%\documentclass[final]{beatcs}
\documentclass[a4paper]{article}
%\documentclass[a4paper,UKenglish]{lipics}


\usepackage{amssymb}
\usepackage{amsmath}
\usepackage{amsthm}
%\usepackage{epigraph}
\usepackage{url}
\usepackage{tikz}
%\usepackage{graphics}
\usepackage{color} 
%\usepackage{fullpage}
\usepackage{graphics}
\usepackage{enumerate}
\usepackage{microtype}
\usepackage{subcaption}
\usepackage{authblk}
\usepackage{a4wide}

\usepackage{thmtools,thm-restate}

%\usepackage[pdftex]{graphicx}
%\ifx\pdftexversion\undefined
%\usepackage[dvips]{graphicx}%\else
%  \usepackage[pdftex]{graphicx}
%  \DeclareGraphicsRule{*}{mps}{*}{}
%\fi

\usepackage[boxed, ruled,noend,linesnumbered]{algorithm2e}

\usepackage{wrapfig}
\usepackage{cite}

%\usepackage[bottom=1in,top=1in,left=1in,right=1in]{geometry}

%%%%%%%%%%%%%%%%%%%%%%%%%%%%%%%%%%%%%%%%%%%%%%%%%%%%%%%%%%%%%%%%%%%%%%%%%=
%%%%%%%%%%%%%%%%%%%%%%%%%%
%pour avoir plus de place
% \advance \topmargin by -\headheight
% \advance \topmargin by -\headsep
%\textheight 8.9in
% \oddsidemargin 0pt
% \evensidemargin \oddsidemargin
% \marginparwidth 0.5in
%\textwidth 6.5in 
% \setlength{\baselineskip}{13.2pt} % standard value is 13.75pt
\def\algofont{\footnotesize} %fonte pour les algos


%fin de pour avoir plus de place
%du coup il faut commenter la suite
%\textwidth 130mm 
%\textheight 215mm
%jusqu'ici
%\renewcommand{\baselinestretch}{1.5}


%Controlling floats
%\renewcommand{\textfraction}{0.07}
%\renewcommand{\floatpagefraction}{0.9}

%%%%%%%%%%%%%%%%%%%%%%%%%%%%%%%%%%%%%%%%%%%%%%%%%%%%%%%%%%%%%%%%
% Definitions
%%%%%%%%%%%%%%%%%%%%%%%%%%%%%%%%%%%%%%%%%%%%%%%%%%%%%%%%%%%%%%%%

\newtheorem{theorem}{Theorem}
% \newtheorem{acknowledgement}[theorem]{Acknowledgement}
% \newtheorem{algorithm}[theorem]{Algorithm}
% \newtheorem{axiom}[theorem]{Axiom}
% \newtheorem{case}[theorem]{Case}
%\newtheorem{claim}[theorem]{Claim}
%\newtheorem{fact}[theorem]{Fact}
% \newtheorem{conclusion}[theorem]{Conclusion}
% \newtheorem{condition}[theorem]{Condition}
% \newtheorem{conjecture}[theorem]{Conjecture}
\newtheorem{corollary}[theorem]{Corollary}
% \newtheorem{criterion}[theorem]{Criterion}
\newtheorem{definition}{Definition}
% \newtheorem{example}[theorem]{Example}
% \newtheorem{exercise}[theorem]{Exercise}
\newtheorem{lemma}[theorem]{Lemma}
% \newtheorem{notation}[theorem]{Notation}
% \newtheorem{problem}[theorem]{Problem}
%\newtheorem{proposition}[theorem]{Proposition}
\newtheorem{property}[theorem]{Property}
% \newtheorem{remark}[theorem]{Remark}
% \newtheorem{solution}[theorem]{Solution}
% \newtheorem{summary}[theorem]{Summary}
%\newenvironment{proof}[1][Proof]{\noindent\textbf{#1.} }{\hfill $\Box$} %\rule{0.5em}{0.5em}\\}
%\newenvironment{proofsketch}[1][Proof sketch]{\noindent\textbf{#1.} }{\hfill $\Box$\\[2mm]}
%\rule{0.5em}{0.5em}\\}
\def\lf{\tiny}
%\def\rrnnll{\setcounter{linenumber}{0}}
%\def\nnll{\refstepcounter{linenumber}\lf\thelinenumber}

%\newcommand{\comment}[1]{\hspace{5mm}\{ \it #1~\}}

\newcommand{\myparagraph}[1]{\vspace{6pt}\noindent \textbf{#1}}

%\newcounter{linenumber}
\newenvironment{algorithmfig}{\hrule\vskip 3mm}{ \vskip 3mm \hrule }

\newcommand{\remove}[1]{}
\newcommand{\ignore}[1]{}




\def\P{\ensuremath{\mathcal{P}}}
\def\DP{\ensuremath{\Diamond\mathcal{P}}}
\def\DS{\ensuremath{\Diamond\mathcal{S}}}
%\def\T{\ensuremath{\mathcal{T}}}
\def\Time{\mathbb{T}}
\def\S{\ensuremath{\mathcal{S}}}
\def\D{\ensuremath{\mathcal{D}}}
\def\W{\ensuremath{\mathcal{W}}}
\def\A{\ensuremath{\mathcal{A}}}
\def\B{\ensuremath{\mathcal{B}}}
\def\F{\ensuremath{\mathcal{F}}}
\def\R{\ensuremath{\mathcal{R}}}
\def\N{\ensuremath{\mathcal{N}}}
\def\I{\ensuremath{\mathcal{I}}}
\def\O{\ensuremath{\mathcal{O}}}
\def\Q{\ensuremath{\mathcal{Q}}}
\def\K{\ensuremath{\mathcal{K}}}
\def\L{\ensuremath{\mathcal{L}}}
\def\M{\ensuremath{\mathcal{M}}}
\def\V{\ensuremath{\mathcal{V}}}
\def\E{\ensuremath{\mathcal{E}}}
\def\C{\ensuremath{\mathcal{C}}}
\def\T{\ensuremath{\mathcal{T}}}
\def\X{\ensuremath{\mathcal{X}}}
\def\Y{\ensuremath{\mathcal{Y}}}

\def\SC{\ensuremath{\mathcal{SC}}}


\def\Nat{\ensuremath{\mathbb{N}}}
\def\Om{\ensuremath{\Omega}}
\def\ve{\varepsilon}
\def\fd{failure detector}
\def\cfd{\ensuremath{?\P+\DS}}
\def\afd{timeless}
\def\env{\ensuremath{\mathcal{E}}}
%\def\faulty{unreliable}
\def\bounded{one-shot}
\def\cons{\textit{cons}}
\def\val{\textit{val}}
\def\code{\textit{code}}

%\def\fair{\textit{steady}}
%\def\Fair{\textit{Steady}}
\def\fair{\textit{fair}}
\def\Fair{\textit{Fair}}

\def\HSS{\mathit{csize}}
\def\argmin{\mathit{argmin}}
\def\proper{\mathit{proper}}
\def\content{\mathit{content}}
\def\Level{\mathit{L}}
\def\Blocked{\mathit{Blocked}}
\def\Set{\mathit{Set}}

\def\Awf{\A_{\textit{wf}}}
\def\Atres{\A_{t\textit{-res}}}
\newcommand{\Ares}[1]{\A_{#1\textit{-res}}}
\def\Alg{\textit{Alg}}

\newcommand{\correct}{\mathit{correct}}
\newcommand{\faulty}{\mathit{faulty}}
\newcommand{\infi}{\mathit{inf}}
\newcommand{\live}{\mathit{live}}
\newcommand{\true}{\mathit{true}}
\newcommand{\false}{\mathit{false}}
\newcommand{\stable}{\mathit{Stable}}
\newcommand{\setcon}{\mathit{setcon}}
%\newtheorem{observation}[theorem]{Observation}

\def\s {\mathbf{s}}
\def\t {\mathbf{t}}
\def\f{\mathbf{f}}
\def\Chr{\operatorname{Chr}}
\def\B {\mathcal{B}}
\def\U {\mathcal{U}}
\def\E {\mathcal{E}}
\def\F{\mathcal{F}}
\def\P{\mathcal{P}}
\def\WF{\mathit{WF}}
\def\Cons{\operatorname{Cons}}
\def\Res{\mathit{Res}}
\def\OF{\mathit{OF}}
\def\O {\mathcal{O}}
\def\I {\mathcal{I}}
\def\bCons{\overline{\Cons}}
\def\Bary{\operatorname{Bary}}
\def\ipart{\infty\text{-}\mathit{part}}
\def\minpart{\mathit{min}\text{-}\mathit{part}}
\def\aff{\operatorname{aff}}
\def\slow{\mathit{slow}}
\def\wf{\operatorname{wf}}
\def\id{\operatorname{id}}
\def\cons{\operatorname{cons}}
\def\ord{\operatorname{ord}}
\def\V{\mathcal{V}}
\def\si {\operatorname{si}}
\def\sy {\operatorname{s}}
\def\nat{\operatorname{nat}}
\def\conc{\operatorname{conc}}
\def\tL{\tilde{L}}
\def\of{\operatorname{of}}
\def\Skel {\operatorname{Skel}}
\def\x{\mathbf{x}}
\def\MEM {\mathit{MEM \!}}
\def\adv{\operatorname{adv}}
\def\Car{\mathit{carrier}}

\def\IStwo{\ensuremath{{\texttt{IS}^2}}}
\def\ISone{\ensuremath{{\texttt{IS}^1}}}

%\def\IStwo{\ensuremath{\Chr^2}}
%\def\ISone{\ensuremath{\Chr^1}}

\def\IS{\textit{IS}}

\def\Cont{\textit{Cont}}


\renewcommand\th{^{\text{th}}}
%\newcommand\st{^{\text{st}}}


\def\Nomega{\ensuremath{\neg\Omega}}
\def\Vomega{\ensuremath{\overrightarrow{\Omega}}}

%\newcommand{\id}[1]{\mbox{\textit{#1}}}% for identifiers in code
%\newcommand{\res}[1]{\mbox{\textbf{#1}}}% reserved words

%% Text to skip

%% Adding notes (in bold) 
\newcommand{\pknote}[1]{\textbf{PK: #1}}
\newcommand{\trnote}[1]{\textbf{TR: #1}}
\newcommand{\yhnote}[1]{\textbf{YH: #1}}
\newcommand{\egnote}[1]{\textbf{EG: #1}}

%% ``set consensus power'' or hitting-set size
%\newcommand{\power}[1]{|\mathcal{HS}(\A|_{#1})|}
\newcommand{\power}[1]{\alpha(#1)}

%\pagestyle{plain}

\begin{document}


%\title{On Compact Representations of Non-Compact Models}
%\title{Compact Topology of Shared-Memory Adversaries}
\title{Asynchronous Computability Theorem for Fair Adversaries}
%\titlerunning{Compact Topology of Shared-Memory Adversaries}
%\running{Compact Topology of Shared-Memory Adversaries}

\author[1]{Petr Kuznetsov}
\author[1]{Thibault Rieutord}
\author[3]{Yuan He}
\affil[1]{LTCI, T\'el\'ecom ParisTech, Universit\'e Paris-Saclay, Paris, France\\
  \texttt{\{petr.kuznetsov,thibault.rieutord\}@telecom-paristech.fr}}
%\affil[2]{LTCI, T\'el\'ecom ParisTech, Universit\'e Paris-Saclay, Paris, France\\
% \texttt{thibault.rieutord@telecom-paristech.fr}}
\affil[2]{UCLA, Los Angeles, USA\\
  \texttt{yuan.he@cs.ucla.edu}}

%\authorrunning{P. Kuznetsov, T. Rieutord and Y. He}

%\Copyright{Petr Kuznetsov, Thibault Rieutord and Yuan He}

%\subjclass{C.2.4 - Distributed Systems, F.1.1 - Models of Computation.}

%\keywords{Adversarial models, Affine tasks, Topological characterization.}


%%%%%%%%%%%%%%%%%%%%%%%%%%%%%%%%%%%%%%%%%%%%%%%%%%%%%%%%%%%%%%%%%%%%%%%%%%%%%%%%
\date{}
\maketitle
\begin{abstract}
The paper proposes a simple topological characterization of
a large class of fair  adversarial distributed-computing models via  
\emph{affine tasks}: sub-complexes of the second
iteration of the standard chromatic subdivision.
We show that the task computability of a model in the class is precisely captured by
iterations of the corresponding affine task. 
%An important advantage of this characterization is that,
%While an adversary is in general defined as a \emph{non-compact} set of 
%infinite runs, its affine task  is just a finite subset of runs of the
%$2$-round iterated immediate snapshot model.
%
Our results generalize and improve all previously derived
topological characterizations of distributed-computing models. 
%
%Our results are derived for abstract models represented via recently
%introduced \emph{agreement functions} and apply, besides
%superset-closed adversaries, to symmetric progress conditions of
%Taubenfeld.
%
%We show that the class of affine task we propose has an element
%equivalent, regarding task solvability, to any adversarial model
%but also to any system which has access to any collection of set-consensus
%objects.
%Our results thus imply the first \emph{compact} combinatorial representation of
%generic (non-compact) adversarial models, as well as large class of deterministic
%objects captured by collection of set-consensus objects.
\end{abstract}

\begin{center}
 {\bf Regular and student paper: T. Rieutord and Y. He are
   full-time students}
 \end{center}



%\thispagestyle{empty}
%\newpage

%-\setcounter{page}{1}

\section{Introduction}

Distributed computing is a jungle of models, parameterized by 
types of failures, synchrony assumptions, and employed communication
primitives.
Determining relative computability power of these models (``is model
$A$ more powerful than model $B$'') is an intriguing and important
problem.

This paper deals with a large class of \emph{shared-memory} models in which a set of
\emph{crash-prone} \emph{asynchronous} processes communicate
via invoking operations on a
collection of shared objects. By default, we assume that the shared objects include 
atomic read-write registers.  


\myparagraph{Topology of wait-freedom.} The \emph{wait-free} model of computation~\cite{Her91} 
makes no assumptions on the processes that can crash. 
%so no correct process can be prevented from making progress.
%
Herlihy and Shavit proposed an elegant 
characterization of wait-free (read-write) task computability through the
existence of a specific continuous map from geometrical
structures describing inputs and outputs of a given
task~\cite{HS99}.
%
A task $T$ has a wait-free solution using read-write registers  
%in the \emph{read-write} shared-memory model
if and only if there exists a simplicial, chromatic map from a
\emph{subdivision}  of the \emph{input} simplicial complex to the \emph{output} simplicial complex,
satisfying the specification of $T$.
%
In particular, we can choose this subdivision to be the iterated
\emph{standard chromatic} subdivision %~\cite{HKR14}
(Figure~\ref{Fig:scs}(a)).
The subdivision precisely captures the output complex of the
\emph{immediate snapshot} (IS) task~\cite{BG97}.
By solving the IS task iteratively, where the view obtained in the
current iteration is used as the input value for the next one, we obtain the
\emph{iterated} immediate snapshot (IIS) model.

%\begin{wrapfigure}{r}{0.45\textwidth}
%\center
%\includegraphics[scale=0.40]{Figures/scs_other.pdf}
%\caption{\label{Fig:scs} \small The standard chromatic subdivision of
%  $\s$ in dimension $2$, equivalent to the output complex of the IS task.}
%\end{wrapfigure}

\begin{figure}
\captionsetup[subfigure]{justification=centering}
  \begin{minipage}[b]{.49\linewidth}
    \centering
\includegraphics[scale=0.45]{Figures/scs_other.pdf}
   \subcaption{\footnotesize{$\Chr^1(\s)$, the output complex of IS}}
  \end{minipage}
  \hfill
  \begin{minipage}[b]{.49\linewidth}
    \begin{center}
\includegraphics[scale=0.5]{Figures/1-res.pdf}
     \subcaption{\footnotesize{$T_1$, the affine task of $1$-resilience
     }}
    \end{center}
  \end{minipage}
  \caption{\small{Subsets of the iterated standard chromatic subdivision}}
  \label{Fig:scs}
\end{figure}

The topological characterization of Herlihy
and Shavit~\cite{HS99} gives rise to the Asynchronous Computability
Theorem:

\begin{quote}
A task $T=(\I,\O,\Delta)$, where $\I$ is the input complex, $\O$ is an
output complex, and $\Delta$ is a carrier map from $\I$ to sub-complexes of
$\O$, is wait-free solvable 
if and only if there exists a natural number $\ell$ and a simplicial map
$\phi: \Chr^{\ell}(\I) \rightarrow \O$ carried by $\Delta$
(informally, respecting the task specification $\Delta$).
\end{quote}

The theorem can be interpreted as: the set of wait-free solvable task is precisely the set of tasks
solvable in the IIS model. 
Thus, the ability of (iteratively) solving the IS task allows us
to solve any task in the wait-free model.
Hence, from the task computability perspective, the IS task is a finite (or, as we
explain below,  \emph{compact}) representation of the
wait-free model.  



\myparagraph{Adversaries.}
%
Given that many fundamental tasks are not solvable in the wait-free
way~\cite{BG93b,HS99,SZ00}, more general models were considered.  
The prominent \emph{adversarial} failure model~\cite{DFGT11} 
is defined through a collection $\A$ of process subsets, called \emph{live sets}, and
requires that, in every run of the corresponding \emph{adversarial $\A$-model}, the set of
processes taking infinitely many steps must be a live set.


%[[PK rather to the related work? 
\ignore{
The underlying structure of the live sets of a given adversary $\A$ 
determines its power in solving \emph{colorless} distributed
tasks\footnote{Intuitively, colorless tasks %~\cite{BG93b,BGLR01}
  can be defined without the notion of process identifiers}.
%
The ability of $\A$ to solve colorless tasks is precisely
captured by its \emph{set consensus
  power}~\cite{DFGT11}\footnote{Although the notion is called
  disagreement power in~\cite{DFGT11}, we chose a more informative
  name proposed in~\cite{GK10}.}.
The set consensus power of $\A$, that can be defined through the formula
proposed in~\cite{GK10}, determines the smallest $k$ such that $k$-set consensus can be solved in
assuming $\A$.
%  
Adversaries having the same set consensus power agree on the set
of colorless tasks they solve.

Consider the special case of \emph{superset-closed} adversaries~\cite{Kuz12},
that, intuitively, do not expect any set of processes to
fail and are, thus, closed under the superset operation.
The set consensus power of a superset-closed adversary is its minimum \emph{core size}:
the size of a smallest set of processes that intersects with
every live set.
%
Herlihy and Rajsbaum~\cite{HR12} related the set consensus power of a
superset-closed adversary with the connectivity properties of the
corresponding topological structure. 
They show that the protocol complex of a superset-closed adversary with 
\emph{minimal core size} $c$ is $(c-2)$-connected.
This result, obtained via an iterative application of the Nerve lemma, gives
a combinatorial characterization of superset-closed adversaries, which
is weaker than the characterization of wait-freedom through
the immediate snapshot task.
%
It only applies to colorless tasks, and it does not allow us to express the
adversary in a \emph{compact} way via a specific task whose iterations
give an equivalent model.  
}

%\myparagraph{Compact topology of adversaries.}
%
For example, the $t$-resilient $n$-process model is defined via
an
%superset-closed
adversary $\A_t$ that consists
of all live sets of size $n-t$ or more. 
%
%Its set \emph{consensus power} is $t+1$.
%[[ 
%: any set of size $t+1$ processes intersects
%with every live set.
%
%
Notice that, assuming the conventional ``longest-prefix''
metric~\cite{AS85}, the model is \emph{non-compact}, in particular, it
does not contain its limit points.
All finite prefixes of an infinite run can be in the model,
even though the run (the infinite limit of the sequence of its
ever-extending finite prefixes) is not.
Consider, for example, an infinite ``solo'' run in which exactly one
process takes steps in a $1$-resilient system of three processes.    


%\begin{wrapfigure}{r}{0.45\textwidth}
%\center
%\includegraphics[scale=0.55]{Figures/1-res.pdf}
%\caption{\label{Fig:tres} \small Graphical representation of $T_1$ in dimension $2$.}
%\end{wrapfigure}

Saraph et al.~\cite{SHG16} recently proposed a direct characterization
of $t$-resilient %(general)
task computability via a specific task $T_t$, defined 
as a restriction of the \emph{double} immediate snapshot
task: the output complex of the task is a subcomplex consisting of
\emph{all} simplices of the second degree of the standard chromatic
subdivision of the task's input complex, except the simplices 
adjacent to the $(n-t-1)$-skeleton of the input
complex (Figure~\ref{Fig:scs}(b) describes the $1$-resilient
$3$-process case).

Solving a %\emph{any} (not necessarily colorless)
task $T$ in the  $t$-resilient model is then equivalent to finding a map from
iterations of $T_t$ (starting from the input complex of $T$) to the output complex of $T$.   
%solving $T_t$: by iterating this
%solution we get a model equivalent to $t$-resilience.   
Therefore, $T_t$ is a compact representation of a (non-compact) model
of  $t$-resilience. 


\myparagraph{Topology of fair adversaries.}
In this paper, we present a compact topological characterization
of the large class of \emph{\fair} adversarial models~\cite{KR17}
that subsumes the models of wait-freedom and $t$-resilience. 
%of a
%a large class of adversaries that includes superset-closed ones.
%
We show that a specific task
$\R_{\A}$ captures the task computability of an adversary $\A$.  
The task consists in solving chromatic simplex
agreement~\cite{BG97,HS99} on a specific \emph{subcomplex} of the second
iteration of the standard chromatic  subdivision.
Such tasks are called \emph{affine}~\cite{GKM14-podc,GHKR16}, as
the geometrical representation of their output complexes are
unions of affine spaces (see, e.g., Figure~\ref{Fig:scs}(b)).  

%We show that the affine task $\R_{\A}$ capturing the
%$k$-set-consensus%
%model consists of all simplices of  the second chromatic subdivision, 
%in which at most~$k$ processes \emph{contend} with each other
%(cf. examples of~$\R_1$ and~$\R_2$  for $3$ processes in 
%Figure~\ref{fig:complexes}).

%\myparagraph{Agreement functions and {\fair} adversaries.}
%
%[[PK unclear and unnecessary
\remove{ 
Our characterization is expressed in an abstract way via
\emph{agreement functions}~\cite{KR17}.
%
For each set of \emph{participating} processes (i.e., processes that
took at least one step in the
computation), the agreement function determines the best level of set
consensus that can be achieved if only these processes \emph{participate} in
the computation.
%
The agreement function corresponding to any adversary can be efficiently computed for
any adversary~\cite{GK10,KR17}.
It turns out that agreement functions encode enough information to
characterize the task computability of any {\fair}
adversary~\cite{KR17}.
}
%

%[[PK too technical: to the preliminaries?
\ignore{
Symmetric adversaries (or \emph{symmetric progress
  conditions} introduced by Taubenfeld. 
%
A symmetric adversary does not depend on process identifiers: if
$S\in\A$, then for every set of processes $S'$ such that $|S'|=|S|$,
we have $S'\in\A$.
%
Thus, an $n$-process symmetric adversary can be defined
as a set $L$ of ``levels'' in
$\{1,\ldots,n\}$ so that a set of processes is a live set if and only
if its size is in $L$.
%
For example, the \emph{$k$-obstruction free} adversary, recently
characterized in an affine way in~\cite{GHKR16}, consists of all
sets of  sizes from $1$ to $k$.
}

Our characterization can then be put as a generalization of the
celebrated Asynchronous Computability Theorem (ACT) by Herlihy and Shavit~\cite{HS99}:
\begin{quote}
A task $T=(\I,\O,\Delta)$ is solvable in a fair adversarial $\A$-model
if and only if there exists a natural number $\ell$ and a simplicial map
$\phi: \R_{\A}^{\ell}(\I) \rightarrow \O$ carried by $\Delta$.
\end{quote}

This result generalizes all topological characterizations of
distributed computing models~\cite{HS99,GKM14-podc,GHKR16,SHG16},
as it applies to \emph{all} tasks (not only colorless)
and \emph{all} fair adversaries (not only
$t$-resilient and $k$-obstruction-free).
Furthermore, our characterization is compact: we match (potentially complicated and non-compact)
fair adversarial models with simple finite affine tasks, defined as 
sub-complexes of the second-degree standard chromatic subdivisions.

Given that there are only finitely many such affine tasks, we conclude
that there can only be finitely many equivalence classes of fair
adversarial models. We believe that the results can be extended to
all ``practical'' restrictions of the wait-free model of computations, beyond fair
adversaries, which may potentially result in a complete computability
theory for distributed computing.    

\myparagraph{Roadmap.} 
%The rest of the paper is organized as follows. 
Section~\ref{sec:model} gives model preliminaries and
Section~\ref{sec:adv} recalls the definitions of adversaries,
agreement functions, and the states the equivalence of
task computability in a fair adversarial model and the corresponding
$\alpha$-model.  
%, briefly overviews the topological representation of iterated
%shared-memory models, and recalls basic simulation techniques used to
%derive our results. 
%In Section~\ref{sec:adv}, we formally define the notion of an agreement
%function and define agreement function of some known models of computation.  
In Section~\ref{sec:def}, we present the definition of the affine task
$\R_{\alpha}$ corresponding to an $\alpha$-model.
In Section~\ref{sec:R}, we show that $\R_{\alpha}$ can be implemented
in most $\alpha$-model. 
In Section~\ref{sec:alpha}, we show that any task solvable in 
the $\alpha$-model can be solved by iterating $\R_{\alpha}$.
Section~\ref{sec:related} reviews related work and
Section~\ref{sec:disc} concludes the paper.
%
Some proofs and missing details are delegated to the
optional appendix.



%======================
\section{Preliminaries}
\label{sec:model}
%======================

Let $\Pi$ be a system composed of $n$ asynchronous processes, 
$p_1,\ldots,p_n$. We consider two models of communication: 
(1)~\emph{atomic snapshots}~\cite{AADGMS93} and 
(2)~\emph{iterated immediate snapshots}~\cite{BG97,HS99}.

\myparagraph{Communication models.}
%
The atomic-snapshot (AS) memory is represented as a vector of 
shared variables, where each process $p_i$ is associated with a distinct 
position $i$. The memory can be accessed with two operations: \emph{update} and 
\emph{snapshot}. An \emph{update} operation performed by $p_i$ 
modifies the value at position $i$ 
and a \emph{snapshot} returns the current state of the memory.
%The model in which processes only have access to an AS memory is 
%called the AS model.

%\myparagraph{Iterated immediate snapshots.}
%
In the iterated immediate snapshot (IIS) model,
processes proceed through a sequence of independent 
memories~$M_1, M_2,\ldots$. Each memory~$M_r$ is accessed 
by a process with a single \emph{immediate snapshot} 
operation~\cite{BG93a}: the operation performed by~$p_i$ takes a
value~$v_i$ and returns a set~$V_{ir}$ of values submitted by 
the processes (w.l.o.g, we assume that the values of 
different processes are distinct), so that the following 
properties are satisfied: (self-inclusion)~$v_i \in V_{ir}$; 
(containment)~$(V_{ir}\subseteq V_{jr}) \vee (V_{jr}\subseteq V_{ir})$; and 
(immediacy)~$v_i \in V_{jr}$ $\Rightarrow$ $V_{ir}\subseteq V_{jr}$. 


\myparagraph{Protocols, runs and models.}
%
A \emph{protocol} here is a deterministic  distributed automaton that, for each 
local state of a process, stipulates which operation and which 
state transition the process is allowed to perform. 
%We assume here \emph{deterministic} protocols, and
%where only 
%one operation and state transition is allowed in each state. A 
A \emph{run} of a protocol is defined as a possibly infinite 
sequence of alternating states and operations.
A \emph{model} is a set of runs.
%

In the IIS communication model, we assume that processes run the 
\emph{full-information} protocol: the first value each process 
writes is its \emph{initial state}. For each $r>1$, the outcome of 
the immediate snapshot operation on memory $M_{r-1}$ is 
submitted as the input value for the immediate snapshot 
operation on memory~$M_r$. After a certain number of iterations,
a process may gather enough information 
to produce an irrevocable \emph{output} value.
%, i.e., to produce an irrevocable non-$\bot$ 
%output value. A \emph{run} of the IIS communication model is 
%thus a sequence~$V_{ir}$,~$i\in\Nat_n$ and~$r\in\Nat$, 
%determining the outcome of the immediate-snapshot operation for 
%every process~$i$ and each iterated memory~$M_r$.

\myparagraph{Failures and participation.}
%
In an infinite run of the AS model,
%(or the defined below AS model in which $k$-set-consensus objects
%can additionally be accessed),
a process that takes only  finitely many steps
%of the assigned  protocol in a given  run
is called \emph{faulty}, otherwise it is called 
\emph{correct}. We assume that in its first step, a process 
writes its initial state in the shared memory using the \emph{update} operation. If a process 
completed this first step in a given run, it is said to be
\emph{participating}, and the set of participating processes is called
the \emph{participating set}.
Note that in an infinite run, every correct process is participating.
%Note that, since every 
%process writes its initial state in its first step, the initial 
%states of participating processes are eventually known to every 
%process that takes sufficiently many steps.

\remove{
In contrast, the IIS model does not have the notion of a faulty 
process. Instead, a process may appear 
``slow''~\cite{RS12,Gaf98-iis,BGK14}, i.e., be late in accessing 
iterated memories from some point on so that some ``faster'' 
processes do not see them. 
}

\myparagraph{Tasks.}
%
In this paper, we focus on distributed \emph{tasks}~\cite{HS99}. 
A process invokes a task with an \emph{input} value and the task 
returns an \emph{output} value, so that the inputs and the outputs 
across the processes, respect the task 
specification.
Formally, a \emph{task} is defined through a set~$\I$ of input 
vectors (one input value for each process), a set~$\O$ of output 
vectors (one output value for each process), and a total 
relation~$\Delta:\I\mapsto 2^{\O}$ that associates each input 
vector with a set of possible output vectors.
We require that $\Delta$ is a \emph{carrier} map: $\forall \tau,\sigma
\in \I$, $\tau\subset\sigma$: $\Delta(\tau)\subseteq \Delta(\sigma)$.  
An input~$\bot$  denotes a \emph{non-participating} process and an output 
value~$\bot$ denotes an \emph{undecided} process. 
Check~\cite{HKR14} for more details on the definition.

A protocol solves a task $T=(\I,\O,\Delta)$ in a model $M$, 
if it ensures that in every infinite run of~$M$ in which 
processes start with an input vector $I\in\I$, there is a finite
prefix $R$ of the run in which: (1)~decided values form a 
vector $O\in\O$ such that $(I,O)\in\Delta$, and (2)~all 
correct processes decide. 

\myparagraph{Standard chromatic subdivision and IIS.}
%
We use the standard language of \emph{simplicial
  complexes}~\cite{Spanier,HKR14} to give a combinatorial
representation of the IIS model. 
A \emph{simplicial complex} is defined as a set of 
\emph{vertices} and an inclusion-closed set of vertex subsets, 
called \emph{simplices}. The dimension of a simplex is 
the number of its vertices minus one. Any subset of these 
vertices is called a \emph{face}. 

%[[PK the wording is unclear, but the notions are standard  
%A simplicial map is a map between simplicial complexes 
%which maps faces of a simplex to faces of a simplex.
%In particular, a vertex map induce a simplicial map.
%]]

A simplicial complex is \emph{pure} (of dimension $n$) if each of its simplices 
is contained in a simplex of dimension $n$.
A simplicial complex is \emph{chromatic} if it is equipped with 
a \emph{coloring function}---a non-collapsing simplicial map $\chi$ from its vertices to 
the {\em standard $(n-1)$-simplex}~$\s$ of $n$ vertices, in 
one-to-one correspondence with $n$ {\em colors} $1,2, \dots, n$. 
%[[PK unnecessary?
%With some abuse of notation, processes may be referred to 
%by their identifiers and~$\chi$ used to obtain 
%the set of processes associated to a simplex.
%]]
%
%Refer to the extended version of this paper~\cite{GHKR16TR} or~\cite{HKR14} for more details on the 
%formalism.

The \emph{standard chromatic subdivision}~\cite{HS99} of $\s$, denoted
$\Chr\s$ and depicted in Figure~\ref{Fig:scs}(a), is a complex where vertices of $\Chr\s$ are couples $(v, \sigma)$, where~$v$ is a 
vertex of~$\s$ and~$\sigma$ is a face of~$\s$ containing~$v$, and
simplices are sets of vertices
$(v_1,\sigma_1)$,~$\ldots$,~$(v_m,\sigma_m)$  satisfying the
properties of immediate snapshot. 
%
$\Chr\s$ is indeed a \emph{subdivision} of $\s$: in particular, 
its geometric realization is homeomorphic to $|\s|$, the geometric realization of $\s$~\cite{Koz12}.
%
If we \emph{iterate} this subdivision~$m$ times, each time 
applying $\Chr$ to each of the simplices, we 
obtain the~$m^{th}$ chromatic subdivision,~$\Chr^m C$.
$\Chr^m \s$ precisely captures  the $m$-round 
(full-information) IIS model, denoted IS$^m$~\cite{HS99}.

The \emph{carrier} of simplex $\sigma\in\Chr^m\s$ relatively to $\Chr^{m'}\s$, 
with $m'<m$, which we denote as~$\Car(\sigma,\Chr^{m'}\s)$ is 
the smallest simplex $\sigma'\in\Chr^{m'}\s$ such that the geometric 
realization of $\sigma$, $|\sigma|$, is included in $|\sigma'|$. 
For example, for a vertex $v$ in $\Chr^m\s$, 
$\Car(v,\Chr^m\s)$ is the set of all processes \emph{seen} by 
the process~$\chi(v)$ in the corresponding run of IS$^m$.
%possibly through the views of other processes: it 
%See Appendix~\ref{app:topprimer} for more details.
(See Appendix~\ref{app:topprimer} for missing details.)

\myparagraph{Simplex agreement and affine tasks.}
%As we show in this paper, the class of fair adversaries can be captured by an iterated 
%\emph{affine} task~\cite{GKM14-podc}.
In a general simplex agreement task, every process is given, as an
input, a vertex of its color in the standard simplex $\s$, and is
expected to output a vertex of $\C$ of its color, so that the outputs
form a simplex of $\C$ respecting the inputs.   
In the instances of simplex agreement considered in characterizations
of wait-free task computability~\cite{BG97,HS99}, inputs were vertices
of $\s$ and $\C$ was a chromatic subdivision of $\s$.
%
Affine tasks can be seen as a generalization of simplex agreement
tasks considered in~\cite{BG97,HS99}, where the output 
complex is no longer a subdivision but a subset of some 
iteration of the standard chromatic subdivision.

Formally, let $L$ be a pure subcomplex of $\Chr^l\s$ for some~$l\in \mathbb{N}$.
The affine task associated to $L$ is then defined as $(\s,L,\Delta)$, 
where, for every face 
$\t \subseteq \s$, $\Delta(\t) = L \cap \Chr^l \t$.
Notice that $L \cap \Chr^l(\t)$ can be empty, in which case no
participating process is required to output.

With a slight abuse of 
notations, we use $L$ to denote the affine task 
associated to $L$. 
By running~$m$ iterations of this task, we obtain $L^m$, a 
subcomplex of $\Chr^{lm}\s$, corresponding to a subset of 
$\IS^{~lm}$ runs (each of the $m$ iterations includes $l$ IS rounds). We 
denote by $L^*$ the set of infinite runs of the IIS model where 
every prefix restricted to a multiple of $l$ IS rounds 
belongs to the subset of $\IS^{~lm}$ runs associated to $L^m$.

%=====================================
\section{Adversaries and agreement functions}
\label{sec:adv}
%=====================================

An \emph{adversary} $\A$ is a set of subsets of $\Pi$, called \emph{live sets}, $\A\subseteq 2^{\Pi}$.
%
%It is convenient to model patterns in which process failures can occur using the formalism of
%\emph{adversaries}~\cite{DFGT11}. An adversary $\A$ is defined as a set of possible correct process subsets.
%
An AS run is \emph{$\A$-compliant} if the set of processes that are correct in that run
belongs to $\A$. An adversarial $\A$-model is defined as the set of
$\A$-compliant runs. 

%\begin{definition}[Agreement function]~\cite{KR17}
%  \label{def:agr}
Consider  an
adversary $\A$ and a function $\alpha_{\A}: 2^{\Pi} \to \{0,\ldots,n\}$. 
We say that $\alpha$ is an \emph{agreement function} of $\A$-model if
for each $P\in 2^{\Pi}$, in the set of runs of $\A$ in which no process
in $\Pi\setminus P$ participates,
$\alpha_{\A}(P)$-set consensus can be solved,  but $(\alpha_{\A}(P)-1)$-set consensus cannot.
Intuitively, $\alpha_A(P)$ is the best level of set consensus that
can be reached when only processes in $P$ are allowed participate.
By convention, if $P$ does not contain a live set, we set $\alpha_\A(P)=0$.

An \emph{$\alpha$-adaptive set consensus} is 
an abstraction solving a set consensus protocol  
such that at the time $k$ distinct output values are returned
the participating set $P$ is such that $\alpha(P)\geq k$: 
the abstraction takes a value as an input, 
returns a proposed value as an output,  and ensures that at the
time when $k$ distinct output has been returned at least
$P$ processes, with $\alpha(P)\geq k$, are currently participating.   

Let $\mathcal{AF}(\Pi)$ be the set of all agreement functions 
corresponding to some adversarial model in $\Pi$. 
For all $\alpha\in\mathcal{AF}(\Pi)$, we can define a natural model
derived from the agreement function as follows:
\begin{definition}[$\alpha$-model] 
  The \emph{$\alpha$-model} is the set of
  %\emph{$\alpha$-adaptive active-resilient} runs, i.e., all
  runs in which, assuming that the participating set is $P$,
  $\alpha(P)\geq 1$ and  at most $\alpha(P)-1$ participating processes are faulty.
%Note that this implies that there is no run of the $\alpha$-model where the participating set $P$ is such that $\alpha(P)=0$.
\end{definition}

%
%%By convention, if $M$ contains no (infinite) runs with participating set
%%$P$, then  $\alpha(P)=0$.
%, e.g., there is no run in
%$M$ with participating set $P$.
%\end{definition}

An adversary is \emph{superset-closed}~\cite{Kuz12} if each superset of a set of an
element of $\A$ is also an element of $\A$, i.e.,
$\forall S\in \A$, $\forall S'\subseteq \Pi$, $S\subseteq S'$: $S'\in\A$. 
Superset-closed adversaries provide an interesting non-uniform
generalization of the classical $t$-resilience condition~\cite{HR10}:
e.g., the \emph{$t$-resilient} adversary in a
system of $n$ processes consists of all sets of $n-t$ or more processes.

%

%[[PK
%Note that the agreement function of a generic adversary $\A$ is
%defined $\setcon(A|P)$~\cite{GK11}. 
%]]

An adversary $\A$ is \emph{symmetric} if it does not depend on process
identifiers and thus only on the size of live sets: 
$\forall S \in \A$, $\forall S' \subseteq \Pi$, $|S'|=|S|\implies S'\in\A$.
%Symmetric adversaries provides another interesting
%generalization of the classical $t$-resilience condition and
%$k$-obstruction-free progress condition~\cite{GG09} which was previously
%formalized by Taubenfeld as its symmetric progress
%conditions~\cite{Tau10}.

%
%\begin{theorem}%~\cite{HR10,GK10}
%  \label{lem:sc-sym}
For  $P\in 2^{\Pi}$, let $\A|_P$ denote the set of all elements of
$\A$ that are subsets of $P$, and $\HSS(\A|_P)$ denote the size of the
minimal hitting set in $\A|_P$, i.e., the minimal subset of $P$ that
intersects with each element in $\A|_P$.  
It is shown in~\cite{GK10} that the smallest $k$ such that $\A$ 
can solve $k$-set consensus can be computed using a function $\setcon(\A)$,
and thus that $\alpha_{\A}(P)=\setcon(\A|_P)$.
For any superset-closed adversary~$\A$, we have
$\alpha_{\A}(P) = \setcon(\A|_P) = \HSS(\A|_P)$.
Moreover, for any symmetric adversary $\A$, we have
$\alpha_{\A}(P)= \setcon(\A|_P) = |\{k\in\{1,\ldots,|P|\}: \exists S \in \A, |S|=k\}|$~\cite{GK10}.
%\end{theorem}

\myparagraph{Fair adversaries.}
%
Informally, an adversary is \emph{{\fair}}~\cite{KR17} if a subset $Q$ of
participating processes $P$ cannot achieve better set consensus than the
whole set of participants (unless $|Q|$ is smaller than~$\alpha_{\A}(P)$).
The set consensus power of a subset $Q$ of the participating processes $P$
corresponds to the set consensus power of the adversary 
$\A|_{P,Q} = \{S\in\A:(S\subseteq P)\wedge(S\cap Q\neq\emptyset)\}.$
%\begin{definition}\label{def:fair}[{\Fair} adversary]
Therefore, an adversary $\A$ is {\fair} if and only if:
\[\forall P \subseteq \Pi, \forall Q\subseteq P, \setcon(\A|_{P,Q})= min(|Q|,\setcon(\A|_P)){}.\]
%\end{definition}

Superset-closed and symmetric adversaries are fair~\cite{KR17}.
%
%We are going to use the following results concerning {\fair}
%adversaries:
It turns out that the task computability of a fair adversary is
captured precisely by the $\alpha_\A$-model, i.e.,
they both solve \emph{the same} set of tasks
(we say that the models are \emph{equivalent}).

\begin{theorem}~\cite{KR17}
\label{th:adv:task}
For any {\fair} adversary $\A$, a task is solvable in the adversarial
$\A$-model if and only if it is solvable in the $\alpha_\A$-model. % are equivalent regarding task solvability.
\end{theorem} 
%

Theorem~\ref{th:adv:task} was proved using the following result, which
will be instrumental for us too:
\begin{theorem}~\cite{KR17}
\label{read/writeAndConsensus}
For any task $T$, if $T$ is solvable in an $\alpha$-model, then $T$ is solvable in \emph{any} read-write shared 
memory model which can solve the $\alpha$-adaptive set consensus task.
\end{theorem}

\remove{
%
\begin{theorem}~\cite{KR17}
  \label{thm:AdaptiveAgreement}
There exists an \emph{$\alpha$-adaptive}  set consensus algorithm
that, in any run with a participating set $P$,
satisfies the following properties: (\emph{Termination}) All \emph{correct} processes
eventually decide;
%
(\emph{Agreement}) At most $\alpha(P)$ different values are decided,
with $P$ the participation at the earliest time a process decided;
%
(\emph{Validity}) Each decided value has been proposed by some process.
\end{theorem}
}

%===================================
\section{Defining the affine task for an $\alpha$-model}
\label{sec:def}
%===================================

Given an agreement function $\alpha\in\mathcal{AF}(\Pi)$, 
we define the affine task $\R_\alpha$, a subcomplex of  
the second degree of the standard chromatic subdivision $\Chr^2\s$. 
In Sections~\ref{sec:R} and~\ref{sec:alpha}, we show that $\R_\alpha^*$, 
i.e., the model of IIS runs obtained by iterating $\R_\alpha$, is equivalent to 
the $\alpha$-model regarding task solvability.

Two classes of affine tasks were recently defined.
The class $\R_{t-res}$ was introduced in~\cite{SHG16}, with
$\R_{t-res}^*$ equivalent to the $t$-resilient model. 
Similarly, the class $\R_k$ was introduced in~\cite{GHKR16}, 
with $\R_k^*$ equivalent to the $k$-concurrent model.
%
Interestingly, the models of $t$-resilience and $k$-concurrency correspond to 
two ``well-behaved'' sub-classes of $\alpha$-models on opposite sides of the spectrum. 
In a sense, an $\alpha$-model can be seen as 
a combination of resilience and concurrency conditions. 
Our definition of $\R_\alpha$
thus combines the concurrency features, expressed through the notion of
\emph{contention} simplices, and the resilience features, expressed
through the notion of \emph{critical} simplices.


\myparagraph{Contention simplices.}
The carrier of a vertex $v$ in $\Chr\s$ determines the view
process $\chi(v)$ obtains after completing the corresponding run of
immediate snapshot (IS).
%
Thus, the \emph{order} in which the processes obtained their immediate snapshots 
materializes in the resulting simplex in $\Chr\s$ through the inclusion-based order of 
the carriers of their vertices. We can therefore define a partial order 
on vertices of a simplex in $\Chr^2\s$, based on the
\emph{views} the processes obtain in the first IS and the
second IS.
%
For a vertex $v\in\Chr^2\s$, we define $\mathit{View}^1(v)=\Car(v',\s)$ 
with $v'$ such that $(\chi(v)=\chi(v'))\wedge(v'\in \Car(v,\Chr \s))$,
i.e., the first IS view that $\chi(v)$ had
in the corresponding run.
%
Similarly, $\mathit{View}^2(v) = \Car(v,\Chr\s)$, i.e., the second IS view.

Note that the more concurrency a run has,
the less processes are ordered regularly.
To capture the fact that a set of processes
was executed concurrently in a run, we compare vertices'
carriers.
%
We say that a simplex $\delta$ in $\Chr^2\s$ is a $2$-contention
simplex, if any two vertices in $\delta$ have different ordered
$\mathit{View}^1$ and $\mathit{View}^2$.
Let ${\mathit{Cont}_2}$ denote the set of $2$-contention simplices in
$\Chr^2 \s$ defined as follows\footnote{In our recent
  paper~\cite{GHKR16} on the affine task for $k$-concurrency (a
  special case of an $\alpha$-model), we introduced the notion of \emph{contention
    sets}, capturing the sets of processes whose vertices in $\Chr^2\s$ have the same carrier
  in $\s$.}:

\begin{definition}{[$2$-Contention simplices]} 
$\forall\sigma\in \Chr^2 \s:
\sigma \in {\mathit{Cont}_2} \Leftrightarrow \forall v,v' \in \sigma, v\neq v':$
\small{
\[((\mathit{View}^1(v) \subsetneq \mathit{View}^1(v'))\wedge(\mathit{View}^2(v') \subsetneq \mathit{View}^2(v)))
\vee
((\mathit{View}^1(v') \subsetneq \mathit{View}^1(v))\wedge(\mathit{View}^2(v) \subsetneq \mathit{View}^2(v'))){}.\]}
\end{definition}
%
The set of $2$-contention simplices is inclusion-closed: any face 
of a $2$-contention simplex is also a $2$-contention simplex. 
Therefore, we can define  
the \emph{$2$-contention complex} as the set of all $2$-contention simplices.
For example, the $2$-contention simplices of the $3$-process system is 
shown in Figure~\ref{Fig:Contention}.

\begin{figure}
\center
\includegraphics[scale=1.2]{Figures/Contention.pdf}
\caption{\label{Fig:Contention} \small The $2$-contention complex is shown in red in dimension $2$.}
\end{figure}

\remove{
\myparagraph{Properties of the $2$-contention simplices.}
Intuitively, the $2$-contention complex capture the notion
of concurrency because, firstly, it provides sufficiently few 
restrictions such that there an algorithm providing simplices 
in $\Chr^2\s$ exists such that: if $k+1$ processes were executing 
$k$-concurrently, then this set of processes is not provided 
with a $k$-dimensional $2$-contention simplex. Moreover,
it is true even if concurrency restriction is present only 
for the resolution of the second level of subdivision.
Indeed, given $k+1$ processes on any of $\Chr\s$ simplices,
waiting for one of the $k$ processes with smaller carriers
in $\Chr^\s$ to terminate before executing the process with
the largest $\Chr^\s$ output will ensure that processes are 
not provided with a $k$-dimensional $2$-contention simplex as output.

In order to grasp concurrency, the definition of $2$-contention simplices
must also provide enough restrictions in order for a solution to
$k$-set consensus exists for the set of all simplices associated 
to any subset of processes $k+1$ processes which are not 
$2$-contention simplices.  
To show that such a solution exist, consider the vertex map $\mu_Q$
such that $\forall v \in \Chr^2\s, {\mathit{dim}}(v)=0, \chi(v)\in Q$:
\[
\mu_Q(v)= \chi(\mathit{min}(\{\Car(v',\s), 
(v'\in \Car(v,\Chr\s))\wedge({\mathit{dim}}(v')=1)\wedge(\chi(v')\in Q)\})|_Q{}.
\]
The vertex map $\mu_Q$ associate to each vertex in $\Chr^2\s$, 
the smallest set of colors of $Q$ observed in the first level of subdivision
by a process from $Q$ which is itself observed by $v$ in the second subdivision.
The number of distinct value provided by $\mu_Q$ on a simplex is linked
to its faces being $2$-contention simplices according to the following property
(the proof can be found in Appendix~\ref{Appendix:ModelProperties}):

\begin{restatable}{property}{ContentionComposability}{[$2$-Contention Simplices Agreement]:\label{prop:ContentionComposability}}
$\forall Q\subseteq\Pi,\forall \sigma\in\Chr^2\s$:\\
\[\chi(\sigma)\subseteq Q \implies |\mu_P(\sigma)|\leq 
max(\{\sigma'\subseteq\sigma, \sigma'\in{\mathit{Cont}_2}\}){}.\]
\end{restatable}

This property easily implies the relation between 
the restriction on $2$-contention simplices dimensions 
and the ability to solve the desired agreement 
among any set of processes for any affine tasks $A\subseteq\Chr^2\s$:

\begin{corollary}{[$2$-Contention and Set Consensus]:} $\forall Q\subseteq\Pi$ :\\
\[max(\{{\mathit{dim}}(\sigma')-1,\sigma'\subseteq\sigma, \sigma\in A, \chi(\sigma)\subseteq Q, \sigma'\in{\mathit{Cont}_2}\})=k \implies \mu_Q \textrm{\ solves\ } k\textrm{-set\ consensus\ in\ } Q{}.\]
\end{corollary}

\begin{proof}
The vertex map $\mu_Q$ is well defined for all vertices of $Q$.
Indeed As $\Car(\sigma,\s)=\Car(\Car(\sigma,\Chr\s),\s)$, 
$\mu_Q(v)\subseteq \chi(\Car(v',\s))$ with $v'\in \Car(v,\Chr\s)$ 
implies that we have~$\mu_Q(v)\subseteq \chi(\Car(v,\s))$. Thus 
$\mu_Q$ respect carrier inclusion property.
Moreover, the vertex map $\mu_Q$ provides, to every process of $Q$, 
non-empty subsets of $Q$ such that at most $k$ distinct sets are provided, cf. Property~\ref{prop:ContentionComposability}.
Thus any deterministically chosen projection from $2^Q$ to $Q$
provides on any simplex at most $k$ distinct proposed inputs,
hence solves $k$-set consensus among $Q$.
\end{proof}
}


\myparagraph{Critical simplices.}
%
Capturing resilience in an affine task~\cite{SHG16} is about ensuring
that every process is provided with sufficiently large 
views, i.e., that $\Car(v,\s)$ is ``large enough'' for its vertex $v$.
%
%In~\cite{SHG16},
Providing a view $P$ such that 
$\alpha(P)>0$ is sufficient to characterize $t$-resilience.
%
To see why this is the case, one can observe that if the participating
set is $P$, then the number of distinct views $V$ with $\alpha(V)>0$
is exactly $\alpha(P)$.
%
In a general $\alpha$-model, however, it is not necessarily the case.
%the number of
%distinct views $V$ with $\alpha(V)>0$ can be strictly higher than
%$\alpha(P)$.  
%
%In an $\alpha$-model, the view that a process must obtain 
%depends on the set consensus power corresponding to the current set of
%participants.
%
%Intuitively,  if a set of processes $Q$ solves $k$-set consensus then 
%the processes in $Q$ must witness a set of processes $P$ 
%such that $\alpha(P)\geq k$.


The definition of $\R_{\alpha}$ should allows us to determine which
of the processes having views $V$ with $\alpha(V)>0$ 
are \emph{prioritized}, i.e., can be used as ``leaders'' in solving
$\alpha(P)$-set consensus.
%in order to ensure that $k$-concurrent 
%executions will observe a high enough participation.
%
When the participation is $P$ with $\alpha(P)>0$, a leader 
with a view $V$ with $\alpha(V)=\alpha(P)$ should be available.
This is why the number ``leaders'' should scale
according the set-consensus power of the participation.
%[[PK no clue where 2-contention comes from here 

\begin{figure}
\captionsetup[subfigure]{justification=centering}
  \begin{minipage}[b]{.49\linewidth}
    \centering
\includegraphics[scale=1.6]{Figures/critical111.pdf}
   \subcaption{\footnotesize{Critical simplices for the $\alpha$-model with $\alpha(P)=min(|P|,1)$ ($1$-obstruction-freedom)}}
  \end{minipage}
  \hfill
  \begin{minipage}[b]{.49\linewidth}
    \begin{center}
\includegraphics[scale=1.6]{Figures/critical_p1-p2p3-p1p2p3.pdf}
    \subcaption{\footnotesize{Critical simplices of the $\alpha$-model corresponding to the adversary $\A=\{\{p_1\},\{p_2,p_3\},\{p_1,p_2,p_3\}\}$}}
    \end{center}
  \end{minipage}
  \caption{\small{Critical simplices are displayed in orange (with $p_1$ associated to the vertex on the top).}}
  \label{fig:critical_examples}
\end{figure}

But also, the number of set consensus proposals pushed 
by leaders when the participation is $P$
should be smaller than $\alpha(P)$.
Moreover, the leaders must be identifiable to other 
processes, in order for them to know that they have 
been prioritized and are not the result
of a high concurrency.  

To capture this intuition, we introduce the following notion.
A simplex $\sigma \in \Chr \s$ is a \emph{critical simplex} if and only if:
(1)~all its vertices share the same carrier;
and (2)~the set consensus power associated to $\Car(\sigma,\s)$ is 
strictly greater than the set consensus power of $\chi(\Car(\sigma,\s))\setminus \chi(\sigma)$.
%(i.e., processes in the carrier of $\sigma$ minus the processes in $\sigma$). 


\begin{definition}{} $\forall\alpha\in\mathcal{AF}(\Pi),\forall \sigma\in \Chr \s$:
\small{
\[
\mathit{Critical}_\alpha(\sigma) = (\forall v \in \sigma: \Car(v,\s)=\Car(\sigma,\s))\wedge
\left(\alpha(\chi(\Car(\sigma,\s))\setminus \chi(\sigma))<\alpha(\chi(\Car(\sigma,\s)))\right){}.
\]
}
\end{definition}

Intuitively, the definition selects simplices which are identifiable 
by vertices which see them in the second IS because 
the selection is defined on the first subdivision, 
which corresponds to the first~IS. 
Examples of critical simplices for some model of 3-process system are given in Figure~\ref{fig:critical_examples}.

For convenience, given a simplex $\sigma\in\Chr\s$, let $\mathcal{CS}_\alpha(\sigma)$ be 
the set of critical simplices in $\sigma$, 
i.e.,~$\mathcal{CS}_\alpha(\sigma)=\{\sigma'\subseteq\sigma:\sigma'\in\mathit{Critical}_\alpha(\sigma)\}$. 
Moreover, let~$\mathcal{CSM}_\alpha(\sigma)$ be 
the set of vertices of $\sigma$ which belong to some critical simplex in $\sigma$, 
i.e.,~$\mathcal{CSM}_\alpha(\sigma)=\cup_{\sigma'\in\mathcal{CS}_\alpha(\sigma)} \sigma'$.
Similarly, let $\mathcal{CSV}_\alpha(\sigma)$ be the union of all processes observed by
critical simplices in $\sigma$, i.e., 
$\mathcal{CSV}_\alpha(\sigma)= \Car(\mathcal{CSM}_\alpha(\sigma),\s)$.


\myparagraph{Concurrency map.}
We note that a critical simplex $\sigma\in\Chr\s$ is associated to 
a set consensus power, i.e., $\alpha(\chi(\Car(\sigma,\s)))$. 
A critical simplex associated to some $k$-set consensus power 
implies the concurrency level to be smaller than or equal to $k$.
This allow us to define the concurrency level for every simplex in $\Chr \s$ by the concurrency of its critical simplices: 

\begin{figure}
\captionsetup[subfigure]{justification=centering}
  \begin{minipage}[b]{.49\linewidth}
    \centering
\includegraphics[scale=1.6]{Figures/ConcurrencyMap111.pdf}
   \subcaption{\footnotesize{Concurrency map for the $\alpha$-model with $\alpha(P)=min(|P|,1)$. (1-obstruction-freedom)}}
  \end{minipage}
  \hfill
  \begin{minipage}[b]{.49\linewidth}
    \begin{center}
    \includegraphics[scale=1.6]{Figures/ConcurrencyMap_p1-p2p3-p1p2p3.pdf}
    \subcaption{\footnotesize{Concurrency map of the $\alpha$-model corresponding to the adversary $\A=\{\{p_1\},\{p_2,p_3\},\{p_1,p_2,p_3\}\}$.}}
    \end{center}
  \end{minipage}
  \caption{\small{Examples of concurrency maps in two models of $3$-process system, a color set to green corresponding to a concurrency value equal to $2$, orange to $1$, and black to $0$. Note that $p_1$ is the vertex on the top.}}
  \label{fig:concurrencyMap}
\end{figure}

\begin{definition}{[Concurrency map] }$\forall \alpha\in\mathcal{AF}(\Pi),\forall \sigma\in\Chr \s$:
\small{
\[
\mathit{Conc}_\alpha(\sigma) = max(0\cup\{\alpha(\chi(\Car(\sigma',\s))), \sigma'\in\mathcal{CS}_\alpha(\sigma)\}){}.
\]
}
\end{definition}

Examples of concurrency maps for two models in a $3$-process system are shown in Figure~\ref{fig:concurrencyMap}. 


\myparagraph{Affine task $\R_{\alpha}$.} The affine task capturing $\alpha$-models can be simply defined, based on 
(1) the set of critical simplices members, $\mathcal{CSM}_\alpha$, their
view, $\mathcal{CSV}_\alpha$, and the concurrency map, $\mathit{Conc}_\alpha$,
all derived from the definition of critical simplices;
and (2), the definition of the $2$-contention simplices. 

The sub-complex $\R_{\alpha}\subseteq\Chr^2\s$ is defined as follows: 
a $(n-1)$-dimensional  simplex $\sigma\in\Chr^2\s$ 
belongs to $\R_{\alpha}$ if and only if, 
every sub-simplex of $\sigma$ of size $k$ 
which is a $2$-contention simplex and 
which does not contain any critical simplices members from $\Chr \s$,
nor are processes in their view,
has a view in $\Chr^2\s$ 
associated to a concurrency greater than or equal to $k$:

\begin{definition}{[$\R_{\alpha}$] }\label{def:RA}$\forall \alpha\in\mathcal{AF}(\Pi),\forall\sigma\in \Chr^2 \s, {\mathit{dim}}(\sigma)=n-1 : \sigma\in\R_\alpha \Leftrightarrow \forall \theta \subseteq \sigma,\theta'= \Car(\theta,\Chr\s):$
\small{\[ (\theta\in{\mathit{Cont}_2})\wedge 
(\chi(\theta) \cap 
(\chi(\mathcal{CSM}_\alpha(\Car(\sigma,\Chr\s)))
\cup\chi(\mathcal{CSV}_\alpha(\theta')))
 = \emptyset 
 \implies
{\mathit{dim}}(\theta)-1\leq \mathit{Conc_\alpha(\theta')}{}.
\]}

\end{definition}
\begin{figure}
\captionsetup[subfigure]{justification=centering}
  \begin{minipage}[b]{.49\linewidth}
    \centering
	\includegraphics[scale=0.50]{Figures/111.pdf}
    \subcaption{\footnotesize{Affine task for the $\alpha$-model with $\alpha(P)=min(|P|,1)$. (1-obstruction-freedom) \label{fig:1-OF}}}
  \end{minipage}
  \hfill
  \begin{minipage}[b]{.49\linewidth}
    \centering
	\includegraphics[scale=0.50]{Figures/p1-p2p3-p1p2p3.pdf}
    \subcaption{\footnotesize{Affine task of the $\alpha$-model corresponding to the adversary $\A=\{\{p_1\},\{p_2,p_3\},\{p_1,p_2,p_3\}\}$.}}
  \end{minipage}
  \caption{\small{Some examples of affine tasks $\R_\alpha$ in blue (with $p_1$ associated to the vertex on the top).}}
  \label{fig:affineTasks}
\end{figure}

Some examples of affine tasks for some $\alpha$-models 
in a 3 processes system are depicted in Figure~\ref{fig:affineTasks}.


%============================================
\section{From $\alpha$-model to $\R_{\alpha}$} % simulation from adversaries}
\label{sec:R}
%============================================

Let $T$ be any task that can be solved in $\R_{\alpha}^*$, i.e.,
by iterating our affine task $\R_{\alpha}$.
To show that $T$ is solvable with a {\fair} adversary $\A$, we present
an algorithm that,  in the corresponding
$\alpha$-model, \emph{solves} $\R_{\alpha}$, i.e.,        
solves the chromatic simplex agreement task
on the complex $\R_{\alpha}$.
%
By iterating this task solution, we obtain a simulation of
$\R_{\alpha}^*$, which implies a solution to $T$ in the $\alpha$-model.


%Solving $\R_\alpha$ is done by a rather simple algorithm executed in the $\alpha$-model.
In our solution of $\R_{\alpha}$ (Algorithm~\ref{Alg:R_A_resolution}),
every process $p_i$ takes two immediate snapshots ~\cite{BG97},
$\mathit{FirstIS}$ and $\mathit{SecondIS}$, where the input of $\mathit{FirstIS}$ is its identifier and the input of $\mathit{SecondIS}$ is the output of $\mathit{FirstIS}$.
%-
%
By construction, the outputs of this algorithm form a simplex in
$\Chr^2\s$. To ensure that this simplex is in $\R_{\alpha}$,
after finishing $\mathit{FirstIS}$, each process $p_i$ writes the outcome in its dedicated register
$\mathit{IS1}[i]$, and then waits for its turn to proceed to 
$\mathit{SecondIS}$.
%
\remove{
processes execute a first immediate snapshot, 
share their immediate snapshot output and then wait their 
``turn'' before executing the second immediate snapshot.
The structure of the algorithm is quite similar to the one
proposed in~\cite{SHG16}. Indeed, as for the $t$-resilient
case, the $\alpha$-model has a structure allowing to directly
solve $\R_\alpha$ without using simulation techniques as it 
would be the case for a generic adversarial model. The difference
with the solution proposed in~\cite{SHG16} just lies in the
more elaborated conditions which must be satisfied before a 
process is allowed to continue with the second immediate
snapshot. 
}
The corresponding ``waiting'' conditions are given in 
Lines~\ref{Alg:RA:WaitBegin}--\ref{Alg:RA:WaitEnd}.
%
Once $\mathit{SecondIS}$ is completed, $p_i$
writes its outcome in $\mathit{IS2}[i]$. 
%The code for a process $p_i$ is presented in Algorithm~\ref{Alg:R_A_resolution}. 
%Processes uses two shared arrays of single-write multi-readers read/write registers 
%$\mathit{IS1}$ and $\mathit{IS2}$, used once per process 
%to share the output they obtained respectively for 
%the first and second immediate snapshot executions.



\begin{algorithm}
\begin{small}
 \caption{Algorithm solving $R_{\alpha}$ in the $\alpha$-model for process $p_i$.\label{Alg:R_A_resolution}}
\SetKwRepeat{Do}{Do}{While}%
$\mathbf{Shared}$: $\mathit{IS1}[1\dots n] \in \mathcal{P}(\Pi)$, $\mathbf{initially}$ $\emptyset$;
$\mathit{IS2}[1\dots n] \in \mathcal{P}(\Pi)$, $\mathbf{initially}$ $\emptyset$\;
$\mathbf{Local}$: $\mathit{critical}[1\dots n] \in
\{\mathit{true},\mathit{false}\}$,
$\mathbf{initially}$ $\mathit{false}$;
$\mathit{concurrency} \in \mathbb{N}$,
$\mathbf{initially}$ $0$\;
$\mathit{level}[1\dots n]\in\mathcal{P}(\Pi)$\;
\vspace{1em}

	$\mathit{IS1}[i] \leftarrow \mathit{FirstIS}(\mathbf{InitialState})$\;
	\Do{$(\neg critical[i])\wedge(|\{p_j\in\mathit{IS1}[i]: 
	\mathit{IS2}[j]=\emptyset\wedge
	\emptyset\subsetneq \mathit{IS1}[j]\subsetneq\mathit{IS1}[i]\wedge
	\neg \mathit{critical}[j]\}|\geq \mathit{concurrency}
	)$}{\label{Alg:RA:WaitBegin}
	  \ForAll{$j\in \{1,\dots,n\}$}{
                        $\mathit{level}[j] \leftarrow   \{p_k\in \Pi, \mathit{IS1}[k]=\mathit{IS1}[j]\}$\;  
%			$\mathit{critical}[j] \leftarrow \left(\alpha(\mathit{IS1}[j])>\alpha(\mathit{IS1}[j]\setminus\{p_k\in \Pi, \mathit{IS1}[k]=\mathit{IS1}[j]\})\right)$\;\label{Alg:RA:UpdateCritical}
                      $\mathit{critical}[j] \leftarrow
            \left(\alpha(\mathit{IS1}[j])>\alpha(\mathit{IS1}[j]\setminus\mathit{level}[j])\right)$\;\label{Alg:RA:UpdateCritical}
			
%	    \If{$\left(\alpha(\mathit{IS1}[j])>\alpha(\mathit{IS1}[j]\setminus\{p_k\in \Pi, \mathit{IS1}[k]=\mathit{IS1}[j]\wedge\mathit{IS2}[j]\neq \emptyset\}\right)$}{
	    \If{$\left(\alpha(\mathit{IS1}[j])>\alpha(\mathit{IS1}[j]\setminus\{p_k\in
              \mathit{level}[j], \mathit{IS2}[k]\neq \emptyset\}\right)$}{
          	    $\mathit{concurrency} \leftarrow \textit{max}(\alpha(\mathit{IS1}[j]),\mathit{concurrency})$\;\label{Alg:RA:UpdateConc}
 			}
		}
	}\label{Alg:RA:WaitEnd}
	$\mathit{IS2}[i] \leftarrow \mathit{SecondIS}(\mathit{IS1}[i])$\;		
		
\end{small}
\end{algorithm}

Intuitively, the waiting phase ensures that processes appearing 
in the critical simplices (maintained in the local
boolean array $\mathit{critical}$) have higher priority to
proceed with $\mathit{SecondIS}$.
%
If the members of a critical simplex finishes $\mathit{SecondIS}$, then
the processes with strictly smaller IS1 output are given priority,
unless their number gets below the currently observed
``concurrency level'' (stored in the local variable
$\mathit{concurrency}$). 
%
Otherwise, the process can proceed to $\mathit{SecondIS}$.
%
Notice that in this case all processes with the same IS1 output will
also be able to proceed to $\mathit{SecondIS}$. 


In its waiting phase, $p_i$ periodically goes over all processes
 and, based on their outputs in $IS1$ and
$IS2$, updates $\mathit{critical}$ and $\mathit{concurrency}$.
%
${Concurrency}$ is set to the maximal set
consensus power of the IS1 output of processes in a critical simplex that 
completed their IS2  (Line~\ref{Alg:RA:UpdateConc}).  


%[[PK please check if I did not mess up
\remove{
Boolean array $\mathit{critical}$ stores the set of processes whose vertices in
$\Chr\s$ belong to currently observed critical simplices.
%
Integer $\mathit{concurrency}$ stores the maximal set consensus power associated to
a critical simplex such that all its processes have terminated their
second IS.

the former corresponds to the increasing set of critical simplices members 
and the latter corresponds to the highest set consensus power associated to
a critical simplices such that all its processes have terminated their 
second immediate snapshot.
The set of $\mathit{critical}$ processes is updated for every processes at
line~\ref{Alg:RA:UpdateCritical} by simply checking if the set of processes
with the same first immediate snapshot output form a critical simplex.
Similarly, for each process, it is checked if the set of processes
with the same first immediate snapshot output, but which also terminated their
second immediate snapshot, also forms a critical simplex. 
If this is the case, $\mathit{concurrency}$ is updated to this
critical simplex associated set consensus power, 
if it is greater than the current value of $\mathit{concurrency}$.

With the two local variables $\mathit{critical}$ and $\mathit{concurrency}$ updated,
processes then test if they should continue to wait 
or are allowed to proceed to the second immediate snapshot (at line~\ref{Alg:RA:WaitEnd}). 
The simplest condition to be allowed to proceed 
consists in being a critical set member, i.e., $\mathit{critical}[i]=\mathit{true}$. 
If a process is not a critical set member, 
then it waits until the number of processes included in 
its first immediate snapshot output (i.e., $p_j\in\mathit{IS1}[i]$) which, 
are not members of a critical simplices (i.e., $\mathit{critical}[j]=\mathit{false}$),
don't have the same first immediate snapshot output 
(i.e., $\mathit{IS1}[j]\subsetneq\mathit{IS1}[i]$),
and have not completed their second immediate snapshot (i.e., $\mathit{IS1}[j]=\emptyset$), 
is strictly smaller than its updated level of $\mathit{concurrency}$.
Only once one of theses two conditions is satisfied, 
a process is allowed to proceed to its second immediate snapshot.	
}
%]]

\begin{restatable}{theorem}{RalphaThm}
For all $\alpha\in\mathcal{AF}(\Pi)$, Algorithm~\ref{Alg:R_A_resolution} solves task $\R_\alpha$ in the $\alpha$-model.\label{Thm:RalphaSimulation}
\end{restatable}

To convince ourselves that Algorithm~\ref{Alg:R_A_resolution} indeed solves
$\R_\alpha$ in the $\alpha$-model, we show that no correct
process can be blocked forever in its waiting phase (Lines~\ref{Alg:RA:WaitBegin}--\ref{Alg:RA:WaitEnd}).
This is because at most $\alpha(P)-1$ processes
can fail and therefore, at least one critical simplex will finish the $\mathit{SecondIS}$,
enabling the remaining processes to proceed. 
%
Moreover, the more faulty processes try to block non-critical 
simplices members with small IS1 to finish their $\mathit{SecondIS}$, the 
less faulty processes are left to block critical simplices 
from finishing their $\mathit{SecondIS}$. But the more critical simplices 
finishes their $\mathit{SecondIS}$, the higher the value of $\mathit{concurrency}$ is,
counter-balancing which processes are faulty.

%[[PK please check if my variant is ok 
\remove{
we first observe that eventually every correct process
either    
%First, it is not obvious to see that every correct process
%will eventually exit the wait-phase and output. This is mostly ensured by the structure
of critical simplices, its members being allowed to continued first before the rest can
follow as most as possible in the same order than in the first immediate snapshot, 
just ensuring that at least one process allowed to continue must be correct (if they
are correct processes left).
}
%]]

The second point we need to ensure is that the resulting simplex of
$\Chr^2\s$ indeed belongs to~$\R_{\alpha}$.
%
One can see this by matching the conditions of the waiting phase with the
definition of $\R_{\alpha}$ (Definition~\ref{def:RA}).  
%
%[[PK same here
\remove{
consists in showing that the second immediate snapshot outputs 
provided to the processes forms in $\Chr^2 \s$ a simplex belonging to $\R_\alpha$.
The fact that it is indeed the case follows mostly from the conditions 
which must be satisfied in order to exit the wait-phase. 
Theses conditions are quite close the restrictions required to form a 
valid $\R_\alpha$ simplex.
}
%]]
The proof of Theorem~\ref{Thm:RalphaSimulation} is given Appendix~\ref{Appendix:RalphaSimulation}.

%===============================================
\section{From $\R_{\alpha}^*$ to $\alpha$-model}
\label{sec:alpha}
%===============================================

In this section, we show that any task solvable in the $\alpha$-model can be solved in $\R_\alpha^*$.
%
More precisely, we present an algorithm that, in
$\R_\alpha^*$, simulates a read-write shared memory system
in which, additionally, $\alpha$-adaptive set consensus can be solved.
%
Then we can use Theorem~\ref{read/writeAndConsensus} to obtain the desired
result.
%
The proofs for this section are delegated to Appendix~\ref{Appendix:alphaSim}.

\myparagraph{General structure of the simulation.}
In principle, our algorithm simulates read-write operations and
accesses to instances $\alpha$-adaptive set consensus (almost) independently.
%
In each round of $\R_{\alpha}^*$, the input of each process to submit to the affine
task $\R_{\alpha}$ is composed of a \emph{read-write} part and  
an \emph{agreement} part. 
%The main difficulty relies in the fact that
%
Even though the simulated algorithm may have only one pending
operation (read-write or propose for a set consensus instance), our algorithm 
requires that the process participates in all currently active agreement instances
and the read-write memory simulation (possibly with ``fake'' inputs
used to help slower processes to complete their simulated operations). 

In general, such a simulation in $\R_{\alpha}^*$  can only be lock-free,
i.e., ensuring that at least one process makes progress, 
as 
%i.e., guaranteeing progress to all 
%processes at the same time. Indeed,
a slow process may never be observed by a fast process in an
iterated model.
%In practice, only lock-free progress can be guaranteed, i.e., guaranteeing 
%that at least one process makes progress.
Lock-freedom is good enough for us, as we are only interested in solving tasks
in the simulated model: once a fast process outputs it ``leaves'' the
computation and does not obstruct slower ones from making progress.
%As we are only
%interested in finite computations, i.e., solving a task,
%this is not an issue as long as progress is guaranteed
%to a process which as not decided a task output yet if 
%one remains.
For this, special input values are  
used by the decided processes to inform slower processes about the task outputs obtained by
the faster ones. 
%know that a task output has been decided and
%that no new memory or agreement operations will be 
%accessed by the process.

%First we show that $\R_{\alpha}$, just one iteration of
%$\R_{alpha}^*$, can be used to solve $\alpha$-adaptive set consensus. 
%
%We first recall a simulation of read-write memory in 
%the iterated snapshot model~\cite{GR10-opodis}. Then 
%we will present an $\alpha$-adaptive set consensus and show how to combine the two
%simulations ($\alpha$-adaptive set consensus and read-write memory) in
%a single lock-free simulation.
%Recall that a lock-free simulation ensures that at least one correct
%process completes all of its operations, which is sufficient for
%simulating any algorithm solving a task in the $\alpha$-model. 

\myparagraph{Read-write simulation.} 
To simulate a read-write shared memory, we rely upon  
the lock-free algorithm proposed in~\cite{GR10-opodis}.
%
This algorithm, conveniently for us, uses iterated, but not necessarily immediate,
snapshots.
%
Thus, we can execute it in the global view resulting of 
iterations of $\R_\alpha^*$, i.e., on $\Car(v,\s)$ for a vertex
$v\in\R_\alpha$. %\pknote{This point (``thus'') is not very clear.} 

%This shared memory simulation is lock-free and also uses 
%processes to terminate by using a special value in order 
%to ensure that an non terminated process makes progress.
To make sure that every active (not yet terminated) participates in
the read-write simulation in every iteration of $\R_{\alpha}^*$, even if the
operations is not required by the simulated algorithm, we make the
process propose the value of its last write operation.
Notice that these ``fake'' inputs do not affect safety of the
simulation, but potentially may affect liveness.
We show that this, however, cannot happen.

%The only difficulty lies in the fact that the algorithm
%always requires non terminated process to have a pending
%memory operation until they terminate. 
%In our case, a process may be non terminated but having 
%a pending agreement operation. To solve this issue, 
%we simply make the processes propose again to write
%the last completed write operation. This does not 
%impact the safety of the simulation, but it may 
%impact the liveness by completing infinitely often
%the same write operation over and over. We will show 
%that this case cannot happend in the general simulation.

An important feature of our algorithm, 
is that, even though the algorithm is lock-free but guarantees that 
if a process completing a simulated read-write operation at the end of a round $r$, 
has the smallest view in $\R_\alpha$ in that round and, thus, every
other process in round $r$ will see the it proposes to a set-consensus instance (if it has one). 
%
Hence,  once a process completes a write, some progress is made:
either it is a write operation of the simulated algorithm, or it is a
``fake'' write, but a simulated access to set
consensus will be completed.      


\myparagraph{Solving $\alpha$-adaptive set consensus in $\R_\alpha^*$.}
%Let us now show how $\alpha$-adaptive set consensus can be solved in $\R_\alpha^*$.
%To do so, we will focus on a single agreement object first and then simply 
%show that any number of such instance can be used independantly.
%
%
In the simulation of accesses to an $\alpha$-adaptive set consensus
instance, every active process adopts a decided or a proposed
value as soon as another process is observed with one.
%We show that the structure of $\R_{\alpha}$ guarantees that the number
%of decided values 
%
%We say that a process is \emph{concerned} by an agreement instance if firstly it 
%is active (i.e., has not been provided with a task output yet), 
%and secondly if it has the right to access the agreement object.

We describe below how the structure of $\R_{\alpha}$ is used to choose
a decided value if no such value is seen.   
%
%Then when every concerned processes which are observed in some iteration
%of $\R_\alpha$ shares a proposal, the structure of $\R_\alpha$ can be 
%used to safely select a decision value for the agreement object. 
%In order to be able to do so, at every iteration, processes must 
%replace their current proposal by a decision estimate based on 
%$\R_\alpha$ structure in order to ensure that if a process selects
%a decision, then the number of possible future decision will 
%satisfy the $\alpha$-adaptive set consensus conditions.
%
One issue to be resolved is that, to avoid potential deadlocks, a process
is expected to take part in an agreement simulation even before it
completes its first simulated write and, thus, before it is considered
participating in the simulated run. 
%The last issue remaining, beside describing the decision estimate
%selection method based on $\R_\alpha$ structure, is that 
%processes are forced to participate to an agreement operation
%even before possibly having completed their first write operation
%making them participating in the simulated run.
Thus, we need to ensure that the number of distinct decided values in
the simulated instance of agreement is allowed by $\alpha$ for the current
participating set.
%
For this, our decision function returns, along with the decided value,
a set of participating processes. Knowing their input is however not
sufficient for these processes to appear participating in the
simulation. This knowledge must be shared with other processes before returning
from the simulated agreement instance. Taking advantage of the read-write
simulation, processes can simply write the inputs of these
processes, which appears in the simulated run as a ``block write.''

%In order to 
%ensure that the participation in the simulated run is high 
%enough compared with the number of possible distinct decided
%value, processes will force other processes to participate 
%before returning their agreement decision. More precisely,
%the selection function returning the decision estimate to 
%select also provides a set of processes which should be 
%participating before returning. Then, to ensure that theses
%processes are participating in the simulated run, it can 
%be simply done by making processes write the initial state
%of every such processes themselves by using the shared %
%memory simulation. 

\remove{
Let us show how to use $\mu_Q$ and ${\mathit{min}}_Q$ to solve an $\alpha$-adaptive set consensus.
We construct a set-consensus protocol among the subset $A$ of
$\Pi$ composed of the processes without a task output.


Consider the following protocol, for a process $v\in A$:
(1) Share the set consensus proposal, if available;
(2) If there exists a process $v'\in\mu_A(v)$ such that 
$v'\in A$ and $v'$ has a set consensus proposal;
Then, (2') adopt the proposal of $min_A$ applied to $\mu_A(v)$
 restricted to the set of processes with a proposal;
Afterwards, (3) If for every process $v'\in\mu_A(v)$ such that 
$v'\in A$, $v'$ has a set consensus proposal;
Then, (3') write the initial state of every process in $\mu_A(v)$
in shared memory(with a simulation presented thereafter); 
(3'') return the proposal of $min_A\circ \mu_A(v)$. 

The first aspect of the set consensus protocol consists in 
waiting for every process without a task output to participate to 
in the set consensus protocol, and to participate itself
as soon as another process with a proposal is 
observed. When every process observed is participating, then
the process first writes the returned participation, $\mu_A(v)$, in 
memory. This 
step is there to ensure that the current participation
is large enough. Note that there is no problem to share
the input value of another process that are not yet participating.
Only then processes return the proposal of $min_A\circ \mu_A(v)$.
}

\myparagraph{Selecting a decision estimate.}
Given $Q\subseteq\Pi$, we introduce a vertex map $\mu_Q$ defined on all vertices
$v\in\R_\alpha$ such that $\chi(v)\in Q$.
Intuitively, $Q$ is the current set of active processes and $\mu_Q(v)$ is
a set of processes (intersecting with $Q$) that should appear participating before the
decided value is returned.
Formally:
%map $\mu_Q$ is defined as follows, for $v\in\R_\alpha, \chi(v)\in Q$:
\begin{small}
\[
\mu_Q(v) = \mathbf{\ if\ } (\chi(\mathcal{CSV}_\alpha(\Car(v,\Chr\s)))\cap Q\neq \emptyset)
\]
\[
\mathbf{then\ } 
\chi(min(\{\Car(\sigma',\s): (\sigma'\in CS_\alpha(\Car(v,\Chr\s)): \chi(\Car(\sigma',\s))\cap Q\neq\emptyset)\})
\]\[
\mathbf{else\ } 
\chi(\mathit{min}(\{\Car(v',\s): 
(v'\in \Car(v,\Chr\s))\wedge({\mathit{dim}}(v')=0)\wedge(\Car(v',\s)\cap Q\neq\emptyset)\}) {}. 
\]
\end{small}

\noindent
We list below several important properties of $\mu_Q$.

%To be valid set of processes expected to be participating, 
%processes must be selected in the set of processes which have been observed
%and thus which the input state is known.
We show first that $\mu_Q$ indeed returns a valid set of processes
with inputs (a subset of the view $v$), which moreover intersects with $Q$:
%\begin{small}
\begin{restatable}{property}{muQVal}{[Validity of $\mu_Q$]\label{muQ_validity}}
$\forall v\in\R_\alpha, {\mathit{dim}}(v)=0, \chi(v)\in Q : $
\[(\mu_Q(v)\subseteq\chi(\Car(v,\s)))\wedge(\mu_Q(v)\cap Q\neq\emptyset){}.\]
\end{restatable}
%\end{small}

\noindent
The principal guarantee $\mu_Q$ needs to ensure is that the number of
distinct sets returned by $\mu_Q$ for a given simplex in $\R_{\alpha}$
cannot exceed the set consensus power of the union of these sets:
%The main property of $\mu_Q$ which will give rise to sufficient 
%agreement for the forced level of participation, concerns the relation between 
%the number of distinct sets returned by $\mu_Q$ and 
%the set consensus power of $\mu_Q$ returned set of processes as follows:
%\begin{small}
\begin{restatable}{property}{muQAg}{[Agreement of $\mu_Q$]\label{muQ_agreement}}
$\forall Q\subseteq\Pi, (\forall \sigma\in\R_\alpha:{\mathit{dim}}(\sigma)= n-1),
(\forall \theta\subseteq\sigma: \chi(\theta)\subseteq Q) :$
\[ |\{\mu_Q(v):v\in\theta\}|\leq \alpha(\cup_{v\in\theta}(\mu_Q(v))){}.\]
\end{restatable}
%\end{small}

\noindent Finally, let us also observe that set $Q$ 
can be determined locally based on the process view, i.e.,
the output of $\mu_Q$ is determined solely by the observed  set of
terminated processes: 
%knowing \emph{all} terminated processes is not necessary and 
%the one observed suffice to obtain the same output for $\mu_Q$:
%\begin{small}
\begin{restatable}{property}{muQRb}{[Robustness of $\mu_Q$]}
$\forall v\in\R_\alpha, {\mathit{dim}}(v)=0, \forall Q\subseteq \Pi:$
\[\mu_Q(v) = \mu_{\Car(v,\s)\cap Q}(v){}.\]
\end{restatable}
%\end{small}

Map $\mu_Q$ can then be used to provide an estimate of the decided
value in an agreement instance.
%A deterministic function mapping to the returned sets by $\mu_Q$ processes
%in $Q$ must be also used.
This is done by selecting any process in the set returned by $\mu_Q$
that belongs to $Q$ and has a proposal value. 
Let ${\mathit{min}}_Q$ be the function returning the minimal such process.   
%minimal element in $Q$ with a proposal value in the set returned by $\mu_Q$.
%
Note that ${\mathit{min}}_Q$ preserves
the robustness property of $\mu_Q$. 
%
It can be shown that if every process has a proposal value, 
then the map ${\mathit{min}}_Q \circ \mu_Q$ defines a decision
map for a set consensus algorithm such that:
%\begin{small}
\begin{restatable}{property}{muQCons}{\label{muQ_set-consensus}}
$\forall Q\subseteq\Pi, (\forall \sigma\in\R_\alpha:{\mathit{dim}}(\sigma)= n-1),
(\forall \theta\subseteq\sigma: \chi(\theta)\subseteq Q): {\mathit{min}}_Q \circ \mu_Q(\sigma)$ solves and $\alpha(\cup_{v\in\theta}(\mu_Q(v)))$-set consensus.
\end{restatable}
%\end{small}

\myparagraph{Combined shared memory and $\alpha$-adaptive set consensus simulation.}
Let us first show that if a "sufficiently fast" process is participating 
in our $\alpha$-adaptive set consensus algorithm, then it must eventually 
terminate:
\begin{restatable}{lemma}{alphaPSC}{\label{AgreementProgress}}
If a process in $A$ with a minimal view in some iteration of $\R_\alpha$ shares a 
proposal, then the described protocol solves $\alpha$-adaptive set consensus.
\end{restatable}

\noindent Note that multiple set consensus protocols can be executed 
in parallel without interfering. 

With such a property, we can now show that the combination of the
read-write memory protocol and the $\alpha$-adaptive set consensus
protocols ensures progress to some active simulated process
as long as there is one:
%
\begin{restatable}{lemma}{MemSCcombined}
The read-write shared memory with access to set consensus simulation is lock-free.
\end{restatable}
%
By Theorem~\ref{read/writeAndConsensus}, the resulting combination of
read-write memory and $\alpha$-adaptive set consensus can be used to solve 
any task solvable in the $\alpha$-model.
Finally, Theorem~\ref{Thm:RalphaSimulation} and
Theorem~\ref{th:adv:task} imply:

\begin{theorem}\label{ref:main}
Let $\A$ be any fair adversary, and let $\alpha$ be its agreement
function.
A task is solvable in the adversarial $\A$-model if and only if it
solvable in  $\R_{\alpha}^*$.
\end{theorem}

As a side result, we get the following generalization of the
Asynchronous Computability Theorem (ACT)~\cite{HS99}:

\begin{theorem}{[Compact ACT]}\label{ref:cact}
Let $\A$ be any fair adversary, and let $\alpha$ be its agreement
function.
A task $T=(\I,\O,\Delta)$ is solvable in the adversarial $\A$-model
if and only if there exists a natural number $\ell$ and a simplicial map
$\phi: \R_{\alpha}^{\ell}(\I) \rightarrow \O$ carried by $\Delta$.
\end{theorem}  


%======================
\section{Related work}
\label{sec:related}
%======================

%\myparagraph{Related work.}
%
%The BGG simulation technique (named after Borowsky, Gafni, and
%Guerraoui) combines ideas of~\cite{BG93a,Gaf09-EBG,GG09,GG11-univ} to
%allow a read-write shared-memory system in which $k$-set consensus can be
%solved to simulated any $k$-process read-write algorithm.
%Details on the technique can be found in~\cite{GHKR16,KR17-TR}.
%
Inspired by the model of \emph{dependent failures}
proposed by Junqueira and Marzullo~\cite{JM07-cores},
Delporte et al.~\cite{DFGT11} suggested the notion of adversaries.
%The calculation of the set consensus power of an adversary
%(called \emph{disagreement power} in~\cite{DFGT11}) used
%in this paper was proposed by Gafni and Kuznetsov in~\cite{GK11}.
Adversaries having the same set consensus power agree on the set
of \emph{colorless} tasks they solve~\cite{DFGT11,GK11}.
Intuitively, a colorless task, such as consensus or approximate agreement, %~\cite{BG93b,BGLR01}
can be defined as a map between a set of inputs and possible sets of
outputs, without the notion of process identifiers.

%The ability of an adversary $\A$ to solve colorless tasks is precisely
%captured by its \emph{set consensus
%  power}~\cite{DFGT11}\footnote{Although the notion is called
%  disagreement power in~\cite{DFGT11}, we chose a more informative
%  name proposed in~\cite{GK10}.}.
%The set consensus power of $\A$, that can be defined through the formula
%proposed in~\cite{GK10}, determines the smallest $k$ such that $k$-set consensus can be solved in
%assuming $\A$.
%  
%Adversaries having the same set consensus power agree on the set
%of colorless tasks they solve.

Herlihy and Shavit~\cite{HS99} proposed a 
characterization of wait-free task computability through the
existence of a specific continuous map from a subdivision 
of the input complex of a task $\I$ to its output complex $\O$.
(The reader is referred to~\cite{HKR14} for a thorough discussion of the use of combinatorial topology 
in distributed computability.)
In particular, the characterization can consider the \emph{iterated} standard
chromatic subdivision and, thus, derive that a task is wait-free
solvable if and only if it can be solved in the IIS model
(capturing precisely by iterated standard chromatic subdivision~\cite{Koz12}).
Thus, the IS task is the affine task matching the wait-free model.
This paper generalizes this result to any {\fair} adversary. 

Herlihy and Rajsbaum~\cite{HR12}
studied colorless task computability
in the special case of \emph{superset-closed} adversaries~\cite{Kuz12},
that, intuitively, do not expect any set of processes 
fail and are, thus, closed under the superset operation.
The set consensus power of a superset-closed adversary is its minimum \emph{core size}:
the size of a smallest set of processes that intersects with
every live set~\cite{JM07-cores}.
%
They show that the protocol complex of a superset-closed adversary with 
\emph{minimal core size} $c$ is $(c-2)$-connected.
This result, obtained via an iterative application of the Nerve lemma, gives
a combinatorial characterization of superset-closed adversaries, which
is weaker than the characterization of wait-freedom through
the immediate snapshot task.
%
Unlike the results of this paper, the characterization only applies to colorless tasks, and it does not allow us to express the
adversary in a \emph{compact} way via a specific task whose iterations
give an equivalent model.  

Gafni et al.~\cite{GKM14-podc} characterized task computability in
\emph{iterated} adversarial models via infinite subdivisions of input
complexes, assuming a limited notion of solvability
that only guarantees outputs to ``fast'' processes~\cite{Gaf98,RS12,BGK14} (''seen'' by every other
process infinitely often).
%
The liveness property defined in this
paper for iterated models, guarantees outputs for \emph{every}
process, which allowed us to establish a task-computability equivalence with 
conventional non-iterated ones. 

Saraph et al.~\cite{SHG16} gave a compact combinatorial
characterization of $t$-resilient  task computability.
Note that $\A_t$ is a superset-closed (and thus {\fair}) adversary.
Our solution of the affine task  $\R_{\alpha}$ in the $\alpha$-model
is inspired by the $t$-resilient solution of $T_t$ in~\cite{SHG16}.   
%
Gafni et al.~\cite{GHKR16} presented affine tasks for the model of $k$-set consensus and, thus,
$k$-concurrency and $k$-obstruction-freedom, which can be expressed as
a symmetric and thus {\fair} adversary.
%

The notions of an agreement function and a fair adversary were introduced by the first two
authors in~\cite{KR17}.
%It has been shown in~\cite{KR17} how to compute
One can determine the agreement function of any given adversary
using the formula suggested earlier for the
set consensus power~\cite{GK10}. 
It has been shown in~\cite{KR17} that agreement functions encode enough information to
characterize the task computability of any {\fair}
adversary.
%

%This paper generalizes all earlier combinatorial characterizations of task
%computability, as it applies to all tasks and all {\fair} adversaries. 


\section{Discussion}
\label{sec:disc}
%
This paper generalizes all topological
characterizations of distributed computing models known so
far~\cite{HS99,GKM14-podc,GHKR16,SHG16}.
It applies to all tasks (not necessarily colorless) and all {\fair}
adversarial models (not necessarily the model of $t$-resilience or $k$-obstruction-freedom).
{\Fair} adversaries include superset-closed~\cite{Kuz12} and symmetric~\cite{Tau10}
ones. 
%
Just as the wait-free characterization~\cite{HS99} implies that the IS task captures the
wait-free model, our characterization equates any {\fair} adversary with
a (compact) affine task embedded in the $2$-degree of standard chromatic subdivision.  

Interestingly, unlike~\cite{SHG16}, we cannot
rely on the assumption that the 
affine task $\R_{\A_t}$ corresponding to the $t$-resilient adversary $\A_t$
is \emph{shellable}~\cite{HKR14} and, thus, link
connected.
%
Link-connectivity of a simplicial complex $\C$ allows us to work in the \emph{point
set} of its geometrical embedding $|\C|$ and use continuous maps (as opposed to
simplicial maps that maintain more structure).
%
For example, the existence of a continuous map from $|\R_{\A_t}|$ to
any $|\R_{\A_t}^k|$ implies that $\R_{\A_t}$ indeed captures the
general task computability of $\A_t$.
%
However, the existence of a continuous map onto $\C$ only allows us to
converge on only \emph{one} vertex~\cite{HKR14}.
If $\C$ is not link-connected, converging on one vertex allow
us to compute only one output in a task solution and not more, which is not
sufficient to solve a general (colored) task.      

%
Unfortunately, only very special adversaries, such as $\A_t$, have
link-connected counterparts (see, e.g., 
the affine task corresponding to $1$-obstruction freedom in Figure~\ref{fig:1-OF}). 
%
Instead, this paper takes an explicit algorithmic way of
showing that iterations of $\R_{A}$ simulate $\A$.   
%
This raises an interesting question to which extent point-set topology
and continuous maps can be applied in affine characterizations.  

Furthermore, going beyond {\fair} models is an important
challenge.
%
Given that some models out of this class cannot be grasped by
agreement functions (some examples of these models can be found
in~\cite{KR17}), we should find a more refined
way of to capture the power of solving set
consensus for subsets of participating processes. In particular, we should be able to account for models
in which \emph{coalitions} of participants can achieve better levels of set
consensus than the whole set.  
%
Nailed down, this may allow us to compactly capture all ``natural''
models~\cite{GHKR16},  such as, e.g., the model of \emph{set
consensus collections}~\cite{DFGK16} for which only special cases of
$k$-set consensus~\cite{GHKR16} and $k$-test-and-set have
been, in this sense, understood so far. 


\bibliographystyle{abbrv}
\bibliography{references}

\newpage

\appendix 

%=======================================
\section{Simplicial complexes}
\label{app:topprimer}
%=======================================

We recall now several notions from combinatorial topology. For more detailed coverage of the topic please refer
to~\cite{Spanier,HKR14}.

A {\em simplicial complex} is a set $V$, together with a collection $C$ of finite non-empty subsets of $V$ such
that:
\begin{enumerate}
\item For any $v \in V$, the one-element set $\{v\}$ is in $C$;
\item If $\sigma \in C$ and $\sigma' \subseteq \sigma$, then $\sigma' \in C$.
\end{enumerate}

The elements of $V$ are called {\em vertices}, and the elements of $C$ are called a {\em simplices}. We usually
drop $V$ from the notation, and refer to the simplicial complex as $C$.

%A simplicial complex $C$ is called {\em finite} if the collection $C$ is finite. A weaker notion is {\em
%locally finite}: $C$ is said to be locally finite if every vertex of $C$ belongs to only finitely many simplices
%in $C$. For simplicity, we will assume that our complexes are locally finite.

A subset of a simplex is called a {\em face} of that simplex.

A {\em sub-complex} of $C$ is a subset of $C$ that is also a simplicial complex.

The {\em dimension} of a simplex $\sigma \in C$ is its cardinality minus one. The $k$-skeleton of a complex $C$,
denoted $\Skel^k C$, is the sub-complex formed of all simplices of $C$ of dimension $k$ or less.

A simplicial complex $C$ is called {\em pure} of dimension $n$ if $C$ has no simplices of dimension $> n$, and
every $k$-dimensional simplex of $C$ (for $k < n$) is a face of an $n$-dimensional simplex of $C$.

%Given a simplex $\sigma \in C$, we denote by $\st \sigma$ the {\em open star} of $\sigma$, that is, the set of
%all simplices in $C$ that have $\sigma$ as a face. The {\em closed star} of $\sigma$, denoted $\St \sigma$, is
%the smallest simplicial complex that contains $\st \sigma$. The difference $(\St \sigma) \setminus (\st \sigma)$
%is called the {\em link} of $\sigma$.

Let $A$ and $B$ be simplicial complexes. A map $f: A \to B$ is called {\em simplicial} if it is induced by a map
on vertices; that is, $f$ maps vertices to vertices, and for any $\sigma \in A$, we have $$ f(\sigma) =
\bigcup_{v \in \sigma} f(\{v\}).$$ A simplicial map $f$ is called {\em non-collapsing} (or {\em
dimension-preserving}) if $\dim f(\sigma) = \dim \sigma$ for all
$\sigma \in A$.

A map $\Phi: A \rightarrow 2^B$ (mapping simplices of $A$ to
sub-complexes of $B$) is called \emph{carrier} if for all $\tau,\sigma
\in A$, we have
$\Phi(\tau\cap\sigma)\subseteq\Phi(\tau)\cap\Phi(\sigma)$.
A simplicial map $\phi: A\rightarrow B$ is said to be \emph{carried by
  a carrier map $\Phi:A\to2^B$} if for all $\sigma\in A$,
$\phi(\sigma) \subset \Phi(\sigma)$.



Any simplicial complex $C$ has an associated {\em geometric realization} $|C|$, defined as follows. Let $V$ be
the set of vertices in $C$. As a set, we let $C$ be the subset of $[0,1]^V = \{\alpha : V \to [0,1]\}$
consisting of all functions $\alpha$ such that $\{ v \in V \mid \alpha(v) > 0 \} \in C$ and $\sum_{v \in V}
\alpha(v) = 1$.
For each $\sigma \in C$, we set $|\sigma| = \{ \alpha \in |C| \mid \alpha(v) \neq 0 \Rightarrow v \in \sigma
\}.$ Each $|\sigma|$ is in one-to-one correspondence to a subset of $\R^n$ of the form $\{(x_1, \dots, x_n) \in
[0,1]^n \mid \sum x_i = 1\}.$ We
put a metric on $|C|$ by $d(\alpha, \beta) = \sum_{v \in V} |\alpha(v) - \beta(v)|.$ 

%Given a simplicial map $f: A \to B$, there is an associated continuous, piecewise linear map $|f|: |A| \to
%|B|$, defined by the formula
%$$ |f|(\alpha)(v') = \sum_{f(v) = v'} \alpha(v).$$

A non-empty complex $C$ is called {\em $k$-connected} if, for each $m\leq k$, any continuous map of the
$m$-sphere into $|C|$ can be extended to a continuous map over the $(m+1)$-disk.

A {\em subdivision} of a simplicial complex $C$ is a simplicial complex $C'$ such that:
\begin{enumerate}
\item The vertices of $C'$ are points of $|C|$.

\item For any $\sigma' \in C'$, there exists $\sigma \in C$ such that $\sigma' \subset |\sigma|$.

\item The piecewise linear map $|C'| \to |C|$ mapping each vertex of $C'$ to the corresponding point of $C$ is a
homeomorphism.
\end{enumerate}

%In particular, every complex $C$ admits a {\em barycentric
%  subdivision} $\Bary(C)$, defined as follows. The vertices of
%$\Bary(K)$ are the barycenters of the simplices of $C$ (in the
%geometric realization). The simplices of $\Bary(K)$ correspond to
%ordered sequences $(\sigma_0, \dots, \sigma_m)$ of simplices of $C$,
%where $\sigma_i$ is a face of $\sigma_{i+1}$; the barycenters of
%$\sigma_i$ are then the vertices of the  corresponding simplex in $\Bary(K)$.

%By iterating this construction $k$ times we obtain the $k$th barycentric subdivision, $\Bary^k(C)$.

\myparagraph{Chromatic complexes.}
%
We now turn to the chromatic complexes used in distributed computing, and recall some notions from \cite{HS99}.

Fix $n \geq 0$. The {\em standard $n$-simplex} $\s$ has $n+1$ vertices, in one-to-one correspondence with $n+1$
{\em colors} $0, 1, \dots, n$. A face $\t$ of $\s$ is specified by a collection of vertices from $\{0, \dots,
n\}$. We view $\s$ as a complex, with its simplices being all possible faces $\t$.
%Note that the open star of a
%face $\t$ is $\st \t = \{\t' \mid \t \subseteq \t' \},$ while the
%closed star of any face is the whole simplex $\s$.

A {\em chromatic complex} is a simplicial complex $C$ together with a non-collapsing simplicial map $\chi: C \to
\s$. Note that $C$ can have dimension at most $n$. We usually drop $\chi$ from the notation. We write $\chi(C)$
for the union of $\chi(v)$ over all vertices $v \in C$. Note that if $C' \subseteq C$ is a sub-complex of a
chromatic complex, it inherits a chromatic structure by restriction.

In particular, the standard $n$-simplex $\s$ is a chromatic complex, with $\chi$ being the identity.

Every chromatic complex $C$ has a {\em standard chromatic subdivision} $\Chr C$. Let us first define $\Chr \s$
for the standard simplex $\s$. The vertices of $\Chr \s$ are pairs $(i, \t)$, where $i \in \{0,1 ,\dots, n\}$
and $\t$ is a face of $\s$ containing $i$. We let $\chi(i, \t) = i$. Further, $\Chr s$ is characterized by its
$n$-simplices; these are the $(n+1)$-tuples $((0,\t_0), \dots, (n, \t_n))$ such that:
\begin{enumerate}[(a)]
\item For all $\t_i$ and $\t_j$, one is a face of the other;
\item If $j \in \t_i$, then $\t_j \subseteq \t_i$. 
\end{enumerate} 
The geometric realization of $\s$ can be taken to be the set $\{\x=(x_0, \dots, x_n) \in [0,1]^{n+1} \mid \sum
x_i = 1\},$ where the vertex $i$ corresponding to the point $\x^i$ with $i$ coordinate $1$ and all others
coordinate $0$. Then, we can identify a vertex $(i, \t)$ of $\Chr \s$ with the point
\[
\frac{1}{2k-1} \x_i + \frac{2}{2k-1} \Bigl( \sum_{\{j \in \t \mid j \neq i\}} \x_j \Bigr) \ \in |\s| \subset
\R^{n+1},
\]
where $k$ is the cardinality of $\t$. 
%(Compare \cite[Definition 5.7]{HS99}.) 
Thus, $\Chr \s$ becomes a subdivision of $\s$ and the geometric realizations are identical: $|\s|=|\Chr \s|$. The standard chromatic subdivision, $\Chr\s$, is illustrated for a 3-process system in Figure~\ref{Fig:scs}(a).

%\begin{wrapfigure}{r}{0.45\textwidth}
%\center
%\includegraphics[scale=0.50]{Figures/scs_other.pdf}
%\caption{\label{Fig:scs} \small The standard chromatic subdivision of $\s$ in dimension $2$.}
%\end{wrapfigure}

Next, given a chromatic complex $C$, we let $\Chr C$ be the subdivision of $C$ obtained by replacing each
simplex in $C$ with its chromatic subdivision. Thus, the vertices of $\Chr C$ are pairs $(p, \sigma)$, where $p$
is a vertex of $C$ and $\sigma$ is a simplex of $C$ containing $p$. If we iterate this process  $m$ times we
obtain the $m\th$ chromatic subdivision, $\Chr^m C$.

Let $A$ and $B$ be chromatic complexes. A simplicial map $f: A \to B$ is called a {\em chromatic map} if for all
vertices $v \in A$, we have $\chi(v) = \chi(f(v))$. Note that a chromatic map is automatically non-collapsing. A
chromatic map has chromatic subdivisions $\Chr^m f: \Chr^m A \to \Chr^m B$. Under the identifications of
topological spaces $|A| \cong |\Chr^m A|, |B| \cong |\Chr^m B|,$ the continuous maps $|f|$ and $|\Chr^m f|$ are identical.

A simplicial map $\phi$ is carried by the carrier
map $\Delta$ if $\phi(\sigma) \subset \Delta(\sigma)$ for every simplex $\sigma$ in their
domain.

%A {\em chromatic multi-map} between $A$ and $B$ is a map $\Delta: A
%\to 2^B$ that, for any $m \leq n$, takes every $m$-simplex of $A$ to
%a pure $m$-dimensional sub-complex of $B$, such that: (i) For every
%simplex $\sigma$ of $A$, we have $\chi(\sigma) =
%\chi(\Delta(\sigma))$, and (ii) For all simplices $\sigma, \tau \in
%A$,  we have
%$ \Delta(\sigma \cap \tau) \subseteq \Delta(\sigma) \cap
%\Delta(\tau).$ In particular, 
%if $\sigma'$ is a face of $\sigma$, then $\Delta(\sigma') \subseteq \Delta(\sigma).$

%============================================
\section{Proof of correctness of Algorithm~\ref{Alg:R_A_resolution}}
%============================================

\label{Appendix:RalphaSimulation}

%\CSAvaibility*

\begin{restatable}{lemma}{CSAvaibility}{[Availability of critical simplices]:\label{lemma:CSAvaibility}}
$\forall\alpha\in\mathcal{AF}(\Pi),\forall\sigma\in\Chr\s:$
\[\chi(\sigma)=\chi(\Car(\sigma)) \implies 
\alpha(\chi(\sigma))\leq \HSS(\mathcal{CS}_\alpha(\sigma)){}.\]
\end{restatable}

\begin{proof} 
We proceed by induction on ${\mathit{dim}}(\sigma)$. 
The base case is trivial, as 
$\HSS(\emptyset)=0$ and $\alpha(\emptyset)=0$.
Now consider a simplex $\sigma\in\Chr\s$ such that 
$\chi(\sigma)=\chi(\Car(\sigma))$ and~$\alpha(\chi(\sigma))= k$, $k\in\mathbb{N}$. 
Let us assume by induction that for all $\sigma'\in \Chr \s$ such that 
${\mathit{dim}}(\sigma')<{\mathit{dim}}(\sigma)$ and $\chi(\sigma')=\chi(\Car(\sigma'))$, we 
have~$\alpha(\chi(\sigma'))\leq \HSS(\mathcal{CS}_\alpha(\sigma'))$.

Now consider the face $\theta$ of $\sigma$ consisting of all vertices of $\sigma$ with the
same carrier as $\sigma$, i.e., $\theta=\{v\in\sigma, \Car(v,\s)=\Car(\sigma,\s)\}$. 
Let $\beta$ be the complement of $\theta$, i.e., $\beta=\sigma\setminus\theta$.
Note that $\theta\neq\emptyset$ and that $\chi(\beta)=\chi(\Car(\beta))$.
Therefore ${\mathit{dim}}(\beta)<{\mathit{dim}}(\sigma)$ and by assumption we have 
$\alpha(\chi(\beta))\leq \HSS(\mathcal{CS}_\alpha(\beta))$.
By the definition of~$\mathcal{CS}_\alpha$, we have 
$\left(\mathcal{CS}_\alpha(\beta)\cup\mathcal{CS}_\alpha(\theta)\right)\subseteq \mathcal{CS}_\alpha(\sigma)$.
Therefore if $\alpha(\chi(\beta))=\alpha(\chi(\sigma))$, we 
have~$\alpha(\chi(\sigma))\leq \HSS(\mathcal{CS}_\alpha(\sigma))$.

Thus, let us assume that $\alpha(\chi(\sigma))=\alpha(\chi(\beta))+l$ with $l>0$.
Note that given any $P\subseteq\Pi$, $P\neq\emptyset$, and any~$p\in P$ we have 
$\alpha(P)\geq\alpha(P\setminus\{p\})\geq\alpha(P)-1$\footnote{Indeed, $\alpha$
is an agreement function derived from an adversary $\A$ and so 
$\alpha(P\cup\{p\})\leq \alpha(P)+1$ as by definition $\setcon(\A|_{P\cup\{p\}})\leq \setcon(\A|_P)+1$ . 
Note that this might not be true for generic $\alpha$ models. Moreover, this 
is the only necessary condition for our equivalence result to hold for generic
$\alpha$ models.}. 
Therefore, by a trivial induction we can deduce that for any set of 
processes $Q\subseteq P$, we have $\alpha(P)\geq\alpha(P\setminus Q)\geq\alpha(P)-|Q|$ 
or equivalently $\alpha(P\setminus Q)+|Q|\geq\alpha(P)$.
So for any subset $\theta'$ of $\theta$ of size ${\mathit{dim}}(\theta)-l+1$, $\chi(\sigma)\setminus\chi(\theta')$ contains $l-1$ more processes than $\chi(\beta)$ and so we have:
\[
\alpha(\chi(\sigma)\setminus(\chi(\theta')))\leq 
\alpha(\chi(\beta))+l-1<
\alpha(\chi(\sigma)){}.
\]
Therefore, $\theta'\in CS_\alpha(\sigma)$. 
Indeed, for every $v\in\theta'$ we have 
$\Car(v,\s)=\Car(\sigma,\s)=\Car(\theta',\s)$.
Moreover, as by assumption we have $\chi(\sigma)=\chi(\Car(\sigma,\s))$,
we obtain $\alpha(\chi(\Car(\theta',\s)))<\alpha(\chi(\Car(\theta',\s))\setminus(\chi(\theta')))$.

This is true for all subsets $\theta'$ of $\theta$ of size ${\mathit{dim}}(\theta)-l+1$.
It is easy to see that $\HSS(\{\theta'\subseteq\theta, {\mathit{dim}}(\theta')={\mathit{dim}}(\theta)-l+1\})=l$, 
and thus $\HSS(CS_\alpha(\theta))\geq l$.
As $\HSS(CS_\alpha(\sigma)) \geq \HSS(CS_\alpha(\theta))+\HSS(CS_\alpha(\beta))$,
we have $\HSS(CS_\alpha(\sigma)) \geq\alpha(\chi(\sigma))$.
\end{proof}

\begin{corollary}{[Distribution of critical simplices]:\label{lemma:CSDistribution}}
$\forall\alpha\in\mathcal{AF}(\Pi),\forall\sigma\in\Chr\s,\forall l\in\{1,\dots,n\}$
\[\chi(\sigma)=\chi(\Car(\sigma)) \implies 
\alpha(\chi(\sigma))-l+1\leq \HSS(\{\sigma'\in\mathcal{CS}_\alpha(\sigma), \alpha(\chi(\Car(\sigma',\s)))\geq l\}){}.\]
\end{corollary}

\begin{proof}
Consider the face $\theta$ of $\sigma$ consisting of 
all vertices of $\sigma$ such that their carrier is associated 
with a set consensus power greater than $\alpha(\chi(\sigma))-l+1$, 
i.e., $\theta = \{v\in \sigma, \alpha(\chi(\Car(v,\s)))\geq l\}$.
It is easy to see 
that~$\{\sigma'\in\mathcal{CS}_\alpha(\sigma), \alpha(\chi(\Car(\sigma',\s)))\geq l\}=\mathcal{CS}_\alpha(\theta)$.

Let $\beta$ be the complement of $\theta$ in $\sigma$, i.e., $\beta=\sigma\setminus\theta$. 
As $\chi(\sigma)=\chi(\Car(\sigma))$, we have $\chi(\beta)=\chi(\Car(\beta))$. 
Therefore, according to Lemma~\ref{lemma:CSAvaibility} we have 
$\alpha(\chi(\beta))\leq \HSS(\mathcal{CS}_\alpha(\beta))$, hence 
$l-1\leq \HSS(\mathcal{CS}_\alpha(\beta))$.
But as $\left(\mathcal{CS}_\alpha(\beta)\cup\mathcal{CS}_\alpha(\theta)\right)=
\mathcal{CS}_\alpha(\sigma)$, we obtain 
$\HSS(\mathcal{CS}_\alpha(\theta))= \HSS(\mathcal{CS}_\alpha(\sigma))-\HSS(\mathcal{CS}_\alpha(\beta))\geq \alpha(\chi(\sigma))-l+1$.
\end{proof}

\RalphaThm*

\begin{proof} We show first that the set of outputs of the processes
  indeed belongs to $\R_\alpha$.
  Then we proceed to the more involved proof of termination.
\\
\noindent \textbf{The outputs belong to $\R_\alpha$: } 
Except for the wait-phase, processes simply execute two rounds of immediate snapshots, 
therefore the set of outputs belongs to some simplex in $\Chr^2\s$. 
In order to belong to $\R_\alpha$, only $2$-contention simplices are restricted. 
Let $\sigma$ be the simplex resulting of the processes outputs. 
Consider a face $\theta$ of $\sigma$ forming a $2$-contention simplex
which does not contain any critical simplex member, i.e.,
$\chi(\theta)\cap\chi(\mathcal{CSM}_\alpha(\Car(\sigma,\Chr\s)))=\emptyset$.

Consider the vertex $v$ in $\theta$, with $i=\chi(v)$, 
which has the smallest carrier $\Car(v,\theta)$. 
Note that vertices in a $2$-contention simplex have distinct
carriers, and so there is a smallest one.
The variable $\mathit{critical}[i]$ can be set to true 
only if the parent of $v$ in the first subdivision, i.e.,
$v'\in\Car(\sigma,\Chr\s)$ such that $\chi(v')=\chi(v)$,
is part in a critical set of $\Car(\sigma,\Chr\s)$.
Indeed, the output of $p_i$ first immediate snapshot
corresponds to $\chi(view^1(v))$, i.e., $\chi(Car(v',\s))$,
and therefore $\mathit{critical}[i]$ was set to 
true only if $v'$ belongs to a critical simplex in $\Car(\sigma,\Chr\s)$.
As $p_i$ exited the wait phase with $\mathit{critical}[i]=\mathit{false}$,
then the following condition was satisfied:
\[|\{p_j\in\mathit{IS1}[i], 
	(\mathit{IS2}[j]=\emptyset)\wedge
	(\emptyset\subsetneq \mathit{IS1}[j]\subsetneq\mathit{IS1}[i])\wedge
	(\neg \mathit{critical}[j])\}|< \mathit{concurrency}{}.
\]
If $\mathit{IS2}[j]\neq\emptyset$, then $p_j$
terminated its second immediate snapshot before $p_i$ started it.
This implies that $p_j$ is associated to a vertex $v_j$ such that
$\Car(v_j,\Chr\s)\subsetneq\Car(v_j,\Chr\s)$ and so $v_j\not\in \theta$.
Moreover, $\emptyset\subsetneq \mathit{IS1}[j]\subsetneq\mathit{IS1}[i]$ 
being false implies, if $j\neq i$, that $View^1(v)\subseteq View^1(v_j)$ 
and thus $v_j\not\in\theta$ as $v$ as $View^2(v)$ is the smallest. 
As $\mathit{critical}[j]$ also implies that $v_j\not\in\theta$,
we obtain that ${\mathit{dim}}(\theta)-1\leq\mathit{concurrency}$ 
(as $p_i$ is not counted in the set) at the time $p_i$ exited the wait-phase.

All that is left to show is that this value of the $\mathit{concurrency}$
variable is smaller than or equal to $\mathit{Conc}(\Car(\theta,\Chr\s)$.
The value of $\mathit{concurrency}$ is set to $k$ if and only if 
a set of processes which terminated their second immediate snapshot
forms a critical simplex in $\sigma$ and has a set consensus power
equal to $k$. Moreover, this critical simplex is a subset of
$\Car(v,\Chr\s)$ as the processes associated to it completed 
their second immediate snapshot. 
Thus, $\mathit{Conc}(\Car(v,\Chr\s)\geq\mathit{concurrency}$.
Moreover as $\Car(v,\Chr\s)\subseteq\Car(\theta,\Chr\s)$, we have
$\mathit{Conc}(\Car(\theta,\Chr\s)\geq\mathit{concurrency}$.
\\
\noindent $\mathbf{Outputs\ are\ provided\ to\ all\ correct\ processes: }$ Assume that the participation is equal to $P$. Therefore there is at most $\alpha(P)-1$ processes which may have crashed in the $\alpha$-model.

Now let us assume that some correct process did not terminate,
and let $p_i$ be such a process with the smallest first immediate
snapshot output.
This means that it fails infinitely often the test on line~\ref{Alg:RA:WaitEnd}, 
and in particular after $\mathit{IS1}$ and $\mathit{IS2}$ were modified for the
last time. Thus the following is always verified for $p_i$:
\[|\{p_j\in\mathit{IS1}[i], 
	(\mathit{IS2}[j]=\emptyset)\wedge
	(\emptyset\subsetneq \mathit{IS1}[j]\subsetneq\mathit{IS1}[i])\wedge
	(\neg \mathit{critical}[j])\}|\geq \mathit{concurrency}{}.\]
As $p_i$ is the correct process with the smallest first immediate snapshot output, 
this set contains only failed processes, as 
$\emptyset\subsetneq \mathit{IS1}[j]\subsetneq\mathit{IS1}[i]$ is true.
Let $k$ be the number of such failed processes. 

If any critical simplex with an associated set consensus power strictly
greater than $k$ was fully simulated in the second immediate snapshot,
then $\mathit{concurrency}$ would be set to a value greater than $k$ and 
$p_i$ would have terminated. But according to 
Corollary~\ref{lemma:CSDistribution}, in any completion
of the first immediate snapshot outputs to include the failed processes,
the minimal hitting set size of the critical sets with associated agreement
power greater than or equal to $k+1$ is greater then or equal to $\alpha(P)-k$.
Thus it requires at least $\alpha(P)-k$ failed processes to block all these 
critical simplices. This implies that the $k$ failed processes blocking $p_i$
and the $\alpha(P)-k$ failed processes blocking the critical simplices 
must intersect. A process in this intersection is not observed by $p_i$ as a critical simplex
member, therefore it cannot have written its first immediate snapshot output.
But if it has not written its first immediate snapshot output, then it is not 
one of the $k$ failed processes blocking $p_i$---a contradiction. 
\end{proof}

\section{Proofs of Section~\ref{sec:alpha}}

For convenience, let us recall the definition of $\mu_Q$, and split it into two vertex maps $\gamma_Q$ and $\delta_Q$ as follows:
\[
\mu_Q(v) = \mathbf{\ if\ } (\chi(\mathcal{CSV}_\alpha(\Car(v,\Chr\s)))\cap Q\neq \emptyset)
\mathbf{\ then\ } \delta_Q
\mathbf{\ else\ } \gamma_Q{}. 
\]
With $\gamma_Q$ and $\delta_Q$ defined as follows:
\begin{small}
\[\delta_Q=\chi(min(\{\Car(\sigma',\s): (\sigma'\in CS_\alpha(\Car(v,\Chr\s)): \chi(\Car(\sigma',\s))\cap Q\neq\emptyset)\}){}.\]
\[\gamma_Q = \chi(\mathit{min}(\{\Car(v',\s): 
(v'\in \Car(v,\Chr\s))\wedge(\mathit{dim}(v')=0)\wedge(\Car(v',\s)\cap Q\neq\emptyset)\}){}.\]
\label{Appendix:alphaSim}
\end{small}

\muQVal*

\begin{proof}
Let us fix some vertex $v\in\R_\alpha$ and some $Q\subseteq\Pi$.
Both $\gamma_Q$ and $\delta_Q$ return the carrier of a subset of the carrier of $v$. 
By definition of the carrier, if $\sigma\subseteq\sigma'$ then $|\sigma|\subseteq |\sigma'|$ and
therefore the carrier of $\sigma$ is a subset of the carrier of $\sigma'$.
Thus, both $\gamma_Q$ and $\delta_Q$ return a subset of $\Car(\Car(v,\Chr\s),\s)$.
By transitivity of inclusion and as $\s$ is a subset of $\Chr\s$,
we have $\Car(\Car(v,\Chr\s),\s)\subseteq\Car(v,\s)$. 
Therefore, $\mu_Q(v)\subseteq\chi(\Car(v,\s))$.

If $\chi(\mathcal{CSV}_\alpha(\Car(v,\Chr\s)))\cap Q\neq \emptyset$ then there is 
a critical set in $\Car(v,\Chr\s)$ which intersects with $Q$. 
Thus $\delta_Q(v)$, and hence $\mu_Q(v)$, returns a set of processes which intersect with $Q$.

Otherwise, $\chi(\mathcal{CSV}_\alpha(\Car(v,\Chr\s)))\cap Q= \emptyset$.
According to the self-inclusion property of immediate snapshots transferred to 
carriers, $\chi(v)\in\Car(v,\Chr\s)$ and thus $\chi(v)\in\mathit{View}^1$. 
Therefore, there is a process $v'\in\Car(v,\Chr\s)$ 
such that $\chi(\Car(v',\s))\cap Q\neq \emptyset$. Therefore $\gamma_Q$, and hence
$\mu_Q$, returns a set of processes intersecting with $Q$.
\end{proof}

Let us show now some auxiliary lemmas that will be used for the proof of Property~\ref{muQ_agreement}:

\begin{lemma}{\label{ContAgreement}}
$\forall Q\subseteq \Pi, (\forall \sigma \in \Chr^2\s: \chi(\sigma)\subseteq Q):$
\[(max(\mathit{dim}(\sigma'): \sigma' \subseteq \sigma \wedge \sigma'\in \mathit{Cont}_2\})=k)
\wedge (\forall v\in\sigma,\exists v'\in\Car(v,\Chr\s),\gamma_Q(v)=\chi(Car(v',\s)))\]
\[ \implies
|\{\gamma_Q(v):v\in\sigma\}|\leq k+1{}.\]
\end{lemma}

\begin{proof}Let us fix some set of processes $Q\subseteq \Pi$ and some simplex $\sigma \in \Chr^2\s$ such that $\chi(\sigma)\subseteq Q$. Consider the vertex $v_1$ of $\sigma$ with one the smallest $View^2(v_1)$ and one the smallest $View^1(v_1)$ among those. According to the definition of $\gamma_Q$ and to the assumption stating that $\gamma_Q$ returns the $View^1(v')$ of a vertex $v'$ of $\sigma$, we can derive that $\gamma_Q(v_1)$ returns $View^1(v_1)$. 

Then let us repeat such an iterative process by selecting the vertex $v_k$ such that $v_k$ has the smallest $View^1(v_k)$ among the processes with the smallest $View^2$ not returning for $\gamma_Q(v)$ the $View^1(v_j)$ of a previously defined $v_j$ with $j<k$. Like before, $v_k$ will return 
$View^1(v_k)$. 

This construction provides a list of vertices of the same size as the number of distinct $\gamma_Q(v)$ outputs in $\sigma$. Moreover, by construction we have $\mathit{View}^2(v_i)\subsetneq \mathit{View}^2(v_j)$ when $i<j$. But if $\mathit{View}^1(v_i)\subseteq\mathit{View}^1(v_j)$, then $v_j$ would not have returned for $\gamma_Q(v_j)$, $\mathit{View}^1(v_j)$. Thus the sets of $v_i$ forms a $2$-contention simplex, proving the lemma claim.
\end{proof}

\begin{lemma}{\label{CSdistribution}}
$\forall \sigma\in \Chr\s,\forall \alpha\in\mathcal{AF}(\Pi),\forall k\in\mathbb{N}:$
\[
|\{\Car(\sigma',\s), \sigma'\in\mathcal{CS}_\alpha(\sigma)\wedge \alpha(\chi(\Car(\sigma',\s)))=k\}|\leq 1{}.
\]
\end{lemma}

\begin{proof}
Let us fix some simplex $\sigma\in \Chr\s$, some agreement function $\alpha\in\mathcal{AF}(\Pi)$ and some $k\in\mathbb{N}$. Assume that there exists two subsets of $\sigma$ which are critical simplices associated with the same set consensus power $k$. Let $C_1$ and $C_2$ be the set of processes composing the critical simplices, and let $V_1$ and $V_2$ be the set of processes corresponding to their carriers.

By assumption we have $V_1\neq V_2$. But by the immediacy and self-inclusion derived from immediate snapshots, we get, w.l.o.g., that $V_1\subseteq (V_2\setminus C_2)$. But according to the definition of critical simplices we have $\alpha(V_2\setminus C_2)<\alpha(V_2)=\alpha(V_1)$ --- A contradiction.
\end{proof}

\muQAg*

\begin{proof}
Let us fix some $Q\subseteq \Pi$, some simplex $\sigma\in\R_\alpha$ such that $\mathit{dim}(\sigma)=n-1$
and some simplex $\theta\subseteq\sigma$ such that $\chi(\theta)\subseteq Q$. 
Let $\{v_1,\dots,v_k\}$ be the vertices of $\theta$.
W.l.o.g, assume that $\Car(v_1,\Chr\s)\subseteq\dots\subseteq \Car(v_k,\Chr\s)$.
According to the test of $\mu_Q$ and the inclusion of the carriers, there is a $l\in\{0,\dots,k\}$,
such that $\forall v_i, i\leq l, \mu_Q(v_i)=\gamma_Q(v_i)$ and $\forall v_i, i> l, \mu_Q(v_i)=\delta_Q(v_i)$. 

Assume that $l=k$. If this is the case,
then none of the vertices of $\theta$ are neither $\mathcal{CSM}_\alpha(v)$, nor in $\mathcal{CSV}_\alpha(v)$. Let $\theta'\subseteq\theta$ be the greatest $2$-contention simplex in $\theta$. According to the definition of $\R_\alpha$ we have $\mathit{dim}(\theta')-1\leq\mathit{Conc}_\alpha(\theta')$. 
Thus there exists a critical set $\sigma_c$, observed by some process in $\theta'$, such that $\mathit{dim}(\theta')-1\leq\alpha(\chi(\Car(\sigma_c,\s)))$. As it is observed by some process in
$\theta'$, and thus in $\theta$, it cannot intersect with $Q$ as otherwise this vertex would have been 
associated to $\delta_Q$. According to carrier inclusion, this implies that the view of every process in
$\theta$ in the first subdivision, i.e., $View^1$, is greater than $\Car(\sigma_c,\s)$.
Therefore, $\alpha(\cup_{v\in\theta}(\mu_Q(v)))\geq {dim}(\theta')-1$. 
By construction the selected processes return views provided by processes from themselves. Thus, according to Lemma~\ref{ContAgreement}, we obtain that 
$|\{\mu_Q(v):v\in\theta\}|\leq \mathit{dim}(\theta')-1 \leq \alpha(\cup_{v\in\theta}(\mu_Q(v)))$

Now let us assume that $l<k$, thus $c = \alpha(min(\{{\Car(\mathcal{CS}}_\alpha(\sigma)):\chi(\Car({\mathcal{CS}}_\alpha(\sigma)))\cap Q\neq\bot\})$ is well defined. 

Now let us consider the processes $v_i$ such that $i\leq l$. As they do not observe a critical set, then
$\gamma_Q$ necessarily maps to $\mathit{View}^1(v)$ with $v$ equal to some $v_i$ such that $i\leq l$.
Let $C$ be the set of processes which return the view of a $v_i\in \mathcal{CS}_\alpha(\sigma)$.
Let us consider the set $V$ of vertices composed of the $v_i$ such that $i> l$ and composed of $C$. For the former, $\mu_Q(v_i)$ returns a set of processes which correspond to the carrier of a critical set of $\sigma$. According to Property~\ref{muQ_validity}, it must be a set which intersects with $Q$. This is also true for the latter according to carrier self inclusion. Therefore by assumption they return the view of a critical set with an associated set consensus power greater or equal to $c$. 
Let $|\{\mu_Q(v),v\in V\}\}|=m$. As $l<k$, we have $m>0$. Then we have a vertex $v_i, i\in\{l+1,\dots,k\}$ such that $\alpha(\mu_Q(v_i))\geq m+c-1$. Indeed, according to Lemma~\ref{CSdistribution} there is at most one carrier of a critical set associated to a given set-consensus value. Therefore if there are $m$ distinct $\delta_Q$ outputs provided, each associated to a set-consensus value greater or equal to $c$, then one is associated to a set consensus power of at least $m+c-1$. Thus $\alpha(\cup_{v\in\theta}(\mu_Q(v)))\geq m+c-1$.

Let $\theta'$ be the greatest $2$-contention simplex composed of the vertices $v_i$ such that $i\leq l$
and such that $v_i\not\in\mathcal{CS}_\alpha(\sigma)$. As they did not pass the test of $\mu_Q$, it implies that they are not in $\mathcal{CSV}_\alpha(\theta')$. Therefore, following the definition of $\R_\alpha$ we have $\mathit{dim}(\theta')-1\leq \mathit{Conc}_\alpha$. The definition implies that  $\mathit{dim}(\theta')-1$ is smaller than the set consensus power of a critical set $\sigma_c$ observed by some process of $\theta'$. As it is observed by some process in $\theta'$, and thus in $\theta$, it cannot intersect with $Q$ as otherwise this vertex would have been associated with $\delta_Q$.
Therefore, it sees a critical set associated to a set consensus power strictly smaller than $c$. 
The construction of the set of processes excludes $V$, the processes returning through $\gamma_Q$ a view of processes which returned $\delta_Q$. Therefore, we can apply Lemma~\ref{ContAgreement} and obtain that there is at most $\mathit{dim}(\theta')-1$
distinct decided values by the process not in $V$. Thus at most $c-1$.

Therefore there are at most $m+c-1$ distinct decided values, and hence:
$|\{\mu_Q(v):v\in\theta\}|\leq m+c-1 \leq \alpha(\cup_{v\in\theta}(\mu_Q(v)))$.
\end{proof}

\muQRb*

\begin{proof}
This is a direct corollary of the definition of $\mu_Q$ and $\gamma_Q$, that a given vertex $v\in\R_\alpha$ does not need $Q$ to compute the result but only the processes in $Q$ it sees.
\end{proof}

\muQCons*

\begin{proof}
Let us consider a vertex $v\in\R_\alpha$,
associated to a process of $Q$, $\chi(v)\in Q$.
According to Property~\ref{muQ_validity}, 
$\mu_Q$ returns to $v$ 
a non-empty subset of processes of $Q$. 
Thus ${\mathit{min}}_Q \circ \mu_Q(v)$ is well defined
and returns a process in $Q$. 
Therefore, the termination property of set consensus is satisfied
for all processes of $Q$.
Moreover, Property~\ref{muQ_validity} also implies that
${\mathit{min}}_Q \circ \mu_Q(v)$ is a subset of $\Car(v,\s)$. 
In combination with the fact that ${\mathit{min}}_Q \circ \mu_Q(v)\in Q$,
this implies that the validity property of set consensus is satisfied
for all processes of $Q$.

Now consider any simplex $\sigma\in\sigma\in\R_\alpha$
composed of processes of $Q$, $\chi(\sigma)\subseteq Q$.
According to Property~\ref{muQ_agreement}, it implies that 
the number of distinct set of processes returned by $\mu_Q$
is smaller than or equal to $\cup_{v\in\theta}(\mu_Q(v))$. Thus
that the number of distinct values returned by ${\mathit{min}}_Q \circ \mu_Q(v)$
is smaller than or equal to $\cup_{v\in\theta}(\mu_Q(v))$.
Therefore ${\mathit{min}}_Q \circ \mu_Q$ also satisfies the 
agreement property of $\cup_{v\in\theta}(\mu_Q(v))$-set consensus.
\end{proof}

\alphaPSC*

\begin{proof}
According to Property~\ref{muQ_set-consensus} $f_A\circ \mu_A$ 
returns a valid set consensus proposal. 
Assume that the protocol does not terminate for some processes
in $A$. This implies that there is a process in $A$ which
is not participating yet. This contradicts the assumption 
that there is a process with the smallest view in 
$A$ in some iteration which participates.

Consider the first round at which some process returns.
As views are inclusion-closed and as the process terminating
must have observed all the processes of $A$ in its view participating,
then every process saw some process in $A$ participating.
Therefore every process adopts a new proposal, which is
$f_A\circ \mu_A(v)$ with $v$ the process associated vertex.
According to Property~\ref{muQ_set-consensus}, we have for all
subset $Q$ of $A$ associated to the simplex $\sigma$ at most 
$\alpha(\cup_{v\in\sigma}(\mu_Q(v)))$ distinct returned
values. But before returning a value, a process 
writes the input state of every process in its $\mu_Q$
output. Therefore the participating set is greater than 
$\cup_{v\in\sigma} \mu_Q(v))$ when $\alpha(\cup_{v\in\sigma}(\mu_Q(v)))$
distinct outputs are returned. Therefore the algorithm 
solves an $\alpha$-adaptive set consensus.
\end{proof}

\MemSCcombined*

\begin{proof}
Assume that eventually the simulation only completes dummy writes. This
means that there is a process which infinitely often completes itself 
dummy writes. Thus this process is infinitely often provided with
the smallest view of $\R_\alpha$ without completing its pending 
set consensus operation. According to Lemma~\ref{AgreementProgress},
every process must have eventually returned --- a contradiction.
\end{proof}

\end{document}
